\section{Conclusion}
In this research, we described the difficulties of integrating distributes RES into 
the existing electric grid and presented Peer-to-Peer (P2P) energy trading in local energy markets (LEM)
as a possible solution to the technical and market problems. Further, we pointed out that central
management in P2P energy trading is challeging due to the need of advanced communication and data exchanges,
wherefore local distributed control and management techniques are more suitable.
Therefore, we introduced the Ethereum-based blockchain as a new and innovative information communication
technology (ICT), which fulfills the need for distributed control and management techniques.
Moreover, we presented a market-based optimization algorithm, which solves a distributed system
optimization problem by self-interested agents that iteratively trading bundled resources in a double
auction market run by a dealer. 
In the end, this research brought all introduced topics together and proposed a concept for a software 
LEM simulation and presented the implementation. 

In the following section, we investigate whether the developed simulation contains all 
components for an efficient operation of a blockchain-based LEM, which are introduced in section \ref{sec:components_of_local_energy_markets}. 
Then, we outline the contribution of this research. Finally, we examine further areas of research.

\subsection{Compliance of LEM Components}
\label{sec:compliance_of_components}

In this subsection we examine if the seven components for an efficient operation of a LEM can be 
provided by the developed simulation platform.
Referring to the microgrid setup (C1), a explicit objective, a definition of the market participants 
and a definition of the form of the traded energy is required. The developed simulation 
is able to meet these requirements. The explicit objective can be defined by the researcher himself by 
setting up a central problem. It is possible to define the market participants and the form of the traded energy bundles
according to your requirements.
Referring to the grid connection (C2), well defined connection points to the superordinate main grid are required.
We have not implemented these connection points. An implementation by an integration into the inventory 
policy of the dealer would be possible. 



\subsection{Contribution}

\subsection{Future Work}

\begin{comment}
    \item[Microgrid setup (C1):] In general, an explicit objective, a definition of the market 
     participants and the form of the traded energy must be well defined. 
     A LEM can have different, often contradictory objectives. Especially in the
     design of the market mechanism, the implementation of the objective plays an important role.
     Next, a significant number of market participants is needed, who trade energy among each other.
     Moreover, a part of the market participants needs to be able to produce energy. 
     Finally, the form of the traded energy must be described, for example electricity, heat or a 
     combination of them. Additionally, the way of energy transportation must be specified.
     Will be the traditional energy grid used or a physical microgrid build. 
    
    \item[Grid connection (C2:)] The connection points to the superordinate main grid 
     must be well defined. These points measure the energy flows towards the main grid 
     and evaluate the performance of the LEM and can help to balance the energy generation 
     and demand within it. Besides, you have to distinguish between a physical microgrid and 
     a virtual microgrid. A physical microgrid bring along a power distribution grid and is able
     decouple from the main grid, whereas a virtual microgrid simply connects all participants over 
     an information system (C3). Consequently, a virtual microgrid does not have the opportunity
     to physically decouple from the main grid. 
     Nevertheless, to operate in island-mode extensively, a physical microgrid need a large 
     amount of own energy generation capacity and flexibility to ensure supply security and robustness.
         
    \item[Information system (C3):] In addition, all participants must be connected to each other
    and a market platform that monitor all operations must be provided. 
    Therefore, a information system is needed, which should enable equal 
    access for every market participant to avoid discrimination. 
    With reference to \shortciteA{mengelkamp2018designing}, these requirements
    can be implemented by blockchain technology based on smart contracts.
    
    \item[Market mechanism (C4):] Besides, a market mechanism implemented through 
     the information system is necessary. This market mechanism implies the allocation of the
     market and the payment rules. Further, a clear bidding format should be defined. 
     It follows that the main objective of the mechanism is to provide an efficient
     energy allocation by matching the buy and sell orders of the participants appropriately.
     Finally, this should happen in near real time granularity.    
    
    \item[Pricing mechanism (C5):] The market mechanism (C4) includes the pricing mechanism 
     and supports the efficient allocation of energy supply and demand. 
     The traditional energy price is composed in large parts of taxes and surcharges.
     On the contrary, in a LEM different fees come to bear, for example in case of a 
     physical microgrid. Hence, RES typically have almost zero marginal cost, prosumer can 
     price their energy above all appropriate taxes and fees to make profit. 
    Thus, the energy price should linked to the availability of energy. In other words,
    a surplus of energy should lower the LEM energy price while a lack of energy inreases the 
    market price. From an economic point of view, LEM are beneficial to their 
    participants as long as the average energy pirce is lower than the external grid price.
        
    \item[Energy management trading system (C6):] The task and goal of the EMTS
     is to automatically ensure energy supply for a respective market participant.
     Therefore, the EMTS needs access to the energy-related data of the participant, like 
     the real time demand and supply. The EMTS uses this data to forecasts consumption and
     generation and develops a bidding strategy accordingly. 
     Moreover, the EMTS trades the predicted amounts on the provided market platform 
     and aims to maximize the revenue and minimize the energy costs. 
     For this reason, the EMTS needs to have access to the market participant’s
     blockchain account to be capable to automatically perform energy transactions.

    \item[Regulation (C7):] It needs to determined how a LEM 
     fit into the current energy policy and which market design is allowed, how 
     taxes and fees are distributed and billed. Likewise, it needs to 
     determined in which way the local market is integrated into the traditional
     energy market and energy supply system.
     All these emerging issues are specified by the legislative regulation
\end{comment}