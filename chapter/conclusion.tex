\section{Conclusion}
In this research, we described the difficulties of integrating distributes RES into 
the existing electric grid and presented Peer-to-Peer (P2P) energy trading in local energy markets (LEM)
as a possible solution to the technical and market problems. Further, we pointed out that central
management in P2P energy trading is challeging due to the need of advanced communication and data exchanges,
wherefore local distributed control and management techniques are more suitable.
Therefore, we introduced the Ethereum-based blockchain as a new and innovative information communication
technology (ICT), which fulfills the need for distributed control and management techniques.
Moreover, we presented a market-based optimization algorithm (BTM), which solves a distributed system
optimization problem by self-interested agents that iteratively trading bundled resources in a double
auction market run by a dealer. 
In the end, we brought all introduced topics together and proposed a concept for a software 
LEM simulation.

Finally, this research presented the technical implementation of the conceptualized
LEM simulation, using the introduced BTM as the market mechanism and 
the Ethereum-based blockchain as the underlying ICT.

In the following section, we investigate whether the developed simulation contains all 
components for an efficient operation of a blockchain-based LEM, which are introduced in section \ref{sec:components_of_local_energy_markets}. 
Then, we outline the contribution of this research. Finally, we examine further areas of research.

\subsection{Compliance of LEM Components}
\label{sec:compliance_of_components}

In this subsection we examine if the seven components for an efficient operation of a LEM,
 elaborated by \shortcite{mengelkamp2018designing}, can be 
provided by the developed simulation platform.
Referring to the microgrid setup (C1), an explicit objective, a definition of the market participants 
and a definition of the form of the traded energy is required. The developed simulation 
is able to meet these requirements. The explicit objective can be defined by the researcher itself by 
setting up a central problem. It is possible to define the market participants and the form of the traded energy bundles
according to your requirements.
Referring to the grid connection (C2), well defined connection points to the superordinate main grid are required.
We have not implemented these connection points. An implementation by an integration into the inventory 
policy of the dealer would be possible. 
As for the information system (C3), the requirements are fulfilled due to the usage of blockchain as the ICT.
Moreover, the applied BTM constitutes an appropriate market mechanism (C4). The BTM defines a clear bidding 
strategy and the overall objective of the applied framework is to provide an optimal allocation in near real-time. 
Additionally, the BTM determines the respective prices by calculating the dual variables.
Therefore, the developed simulation also includes a pricing mechanism (C5) that supports an efficient allocation.
Likewise, a suitable energy management trading system (C6) is provided by the applied BTM. 
It ensures automatically the energy supply and aims to minimize the energy costs.
Finally, the developed simulation lacks the definition of how a LEM fit into 
the current energy policy and how taxes and fees are distributed and billed. 
That is why the component regulation (C7) is not fulfilled. 
In conclusion, the developed simulation provides five of seven components for an efficient operation of a LEM. 
Besides, it allows the integration of the missing two components.
That represents a appropriate and valuable basis for a substantial and significant simulation of a LEM. 

\subsection{Contribution}
The conducted research provides two core contributions:

First, the decentralized implementation of the proposed BTM. 
We successfully implemented the BTM with a blockchain and the included smart contracts 
as the applied ICT. 
Due to the application of the blockchain technology, the technical implementation
is publicly accessible.
Therefore, more transparency is provided and the required level of trust in the BTM
is reduced. Additionally, the blockchain contains cryptographic encryption methods,
which raises the level of security in the BTM.
Concluding, a secure and decentralized market-based optimization algorithm for distributed systems has 
been developed.

Second, we conceptualized a LEM simulation based on the implemented
BTM. We proposed an examplary linear programming problem in terms of energy efficient 
demand side management of household to embed the BTM into the topic of LEM.
Moreover, the conducted research removed the technical barriers of using complex 
distributed ledger technologies for researchers.
Therefore, it incentivises researchers 
to design and test their artifacts by using the developed simulation.
Furthermore, due to the customizability of the blockchain properties 
the developed simulation allows to test different scenarios and thus to get a
better understandig of the dynamics of a decentralized LEM.

\subsection{Future Work}
The conducted research offers potential for future work in many areas.

First, we stated in section \ref{sec:market_clearing_mechanism}, that we
applied a synchronous call market, where submitted orders to the system
will be accumulated and processed simulationously at periodic intervalls.
In a real world application, it would be more appropriate to clear 
the market every time a new orders arrives. Therefore,
the implementation from the synchronous to the introduced ansynchronous market trading 
environment of \shortciteA{guo2012computational} would be reasonable. 

Second, the implementation of an 
auction mechanism by a blockchain causes 
difficulties regarding privacy. Every on-chain data 
is publicly available. 
In case of an auction mechanism, this can result in competitive advantages
for those who submit their orders later on. 
In an environment where all orders submitted at the same time and written 
into the same block, this would not be an issue. 
Otherwise, some participants will have a competitive advantage.
In a real world application, it is more likely that the orders would be submitted at different
times and would not be written into the same block. 
As a result, future work could address the implementation of commitment schemes
in the area of agent bidding to avoid those issues.

Finally, as described in this reasearch, we implemented no strategic
actions in the bundle selection and pricing. 
However, \shortciteA{guo2012computational} represent various 
forms of agent strategic behavior regarding bundle selection 
and bundle pricing.
In their proposals, agents use forecasted market prices 
to determine the prefered bundles
instead of the most recently observed market prices. 
These approaches could be adopted by future work and implemented
into the presented LEM simulation.
In addition, \shortciteA{guo2012computational} also introduce two
different inventory policies of the dealer to examine if 
holding intertemporal inventory can facilitate real-time trades 
in the BTM environment. The implementation of those inventory policies
into the presented simulation and the adaption of these policies
into the context of LEM could be also of interest for future work.  








\begin{comment}

\end{comment}