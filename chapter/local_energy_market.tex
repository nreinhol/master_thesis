\section{Local Energy Markets}
\label{sec:lem}
This section explains the general concept of LEM
and presents possbile market components for the efficient building of LEM.

Due to the centralized generation by large power plants
and the current design of the wholesale markets, the existing grid is not suitably designed
to react in real-time to a significant growing 
amount of distributed RES \shortcite{ampatzis2014local}.
To successfully integrate and use these RES, new approaches are necessary \shortcite{mengelkamp2018blockchain}.
P2P energy trading in LEM provides a possible solution to the technical and conceptual problems \shortcite{long2017feasibility}.
In the traditional centralized energy system, large power plants that operate according to a
centralized coordination mechanism supply a large amount of customers with energy, who are located 
within a wide area (for example a country or a state).
In contrast, decentralized energy systems consist of small-scale energy generators that are 
only used by a small number of people and located close to the energy consumption point \shortcite{mengelkamp2018designing}.
Those LEM provide a market platform, market mechanism, and market access
for small-scale prosumer and consumer, to trade locally generated energy within their community.
A community presents a group of geographically and socially close energy agents.
All participants buy or sell energy directly with each other by using the provided market platform
without intermediation by conventional energy suppliers \shortcite{zhang2017review}.
If prosumers have a surplus in electricity, they have the opportunity 
to curtail it, store it in an energy storage device or export it back to the main power grid \shortcite{zhang2017review}.
In the traditional centralized energy system, trading of energy is mainly unidirectional.
As stated above, electricity is usually transmitted from large-scale power plants to 
consumers over long distances, while the cash flow goes the opposite way. 
On the contrary, the P2P energy trading in LEM encourages multidirectional trading within 
a local geographical community \shortcite{zhang2017review}, as illustrated in Figure \ref{figure:microgrid}.

\begin{figure}[htbp]
	\centering
	\includegraphics[width=.9\linewidth]{./figures/microgrid_1024x.png}
	\caption{Example of Microgrid setup \protect\shortcite{mengelkamp2018designing}}
	\label{figure:microgrid}
\end{figure}

Due to this, LEM promote the consumption of energy close to its generation and, therefore, 
support sustainability and the efficient use of local resources \shortcite{mengelkamp2018designing}.
LEM enable (near) real-time pricing and facilitate a local balance
of supply and demand \shortcite{mengelkamp2018blockchain}. 
As stated by \shortciteA{mengelkamp2018designing},
the participants of LEM do not necessarily have to be physically connected. Virtual LEM
describe the aggregated control of multiple energy producers, prosumers and consumers in a virtual 
community.
Referring to \shortciteA{jimeno2011architecture}, non-virtual LEM have a specific characteristic
that distinguishes it from other aggregation of DER systems. They have the opportunity
to operate either connected to the main grid or islanded from it. In other words,
this allows non-virtual LEM to run disconnected from the main grid, in case it fails or 
the power quality is not satisfactory.
P2P energy trading in LEM requires advanced communication and data 
exchanges between the different parties, which makes central management and 
operation more and more challenging. The implementation of LEM needs 
local distributed control and management techniques \shortcite{andoni2019blockchain}. 
\shortciteA{zhang2017review} state that P2P energy trading is often
enabled by ICT-based online services. Moreover, \shortciteA{mengelkamp2018designing} explain that 
the new and innovative blockchain technology as an emerging ICT, 
offers new opportunities for decentralized market designs.
It is designed to enable distributed transactions without 
a central trusted entity.
Accordingly, blockchain can help to address the challenges faced by 
decentralized energy systems. 

\subsection{Components of Local Energy Markets}
\label{sec:components_of_local_energy_markets}
\shortciteA{mengelkamp2018designing} developed seven components for efficient
operation of blockchain-based LEM. This subsection names and briefly 
illustrates each component. Further, the compliance of BLEMS and the seven components will be examined later on in section \ref{sec:compliance_of_components}.
We note that these elaborated components have no general validity and it could also be 
possible to efficiently implement LEM without providing these components. 
However, we use these components for an initial evaluation of the platform and its capabilities.

\begin{description}
    \item[Microgrid Setup (C1):] In general, an explicit objective, a definition of the market 
     participants and the form of the traded energy must be well defined. 
     LEM can have different, often contradictory objectives. Especially in the
     design of the market mechanism, the implementation of the objective plays an important role.
     Next, a significant number of market participants is needed, who trade energy among each other.
     Moreover, a part of the market participants needs to be able to produce energy. 
     Finally, the form of the traded energy must be described, for example, electricity, heat or a 
     combination of them. Additionally, the way of energy transportation must be specified.
     That means, is the the traditional energy grid or a physical microgrid used for energy transportation.
     
    
    \item[Grid Connection (C2:)] The connection points to the superordinate main grid 
     must be well defined. These points measure the energy flows towards the main grid 
     and evaluate the performance of the LEM and can help to balance the energy generation 
     and demand within it. Besides, you have to distinguish between a physical microgrid and 
     a virtual microgrid. A physical microgrid brings along a power distribution grid and is able to
     decouple from the main grid, whereas a virtual microgrid simply connects all participants over 
     an information system (C3). Consequently, a virtual microgrid does not have the opportunity
     to physically decouple from the main grid. 
     Nevertheless, to operate in island-mode extensively, a physical microgrid need a large 
     amount of own energy generation capacity and flexibility to ensure supply security and robustness.
         
    \item[Information System (C3):] All participants must be connected 
    and a market platform that monitors all operations must be provided. 
    Therefore, an information system is needed, which should enable equal 
    access for every market participant to avoid discrimination. 
    With reference to \shortciteA{mengelkamp2018designing}, these requirements
    can be implemented by blockchain technology based on smart contracts.
    
    \item[Market Mechanism (C4):] A market mechanism implemented through 
     the information system is necessary. This market mechanism determine the allocation of the
     market and the payment rules. Further, a clear bidding format should be defined. 
     It follows that the main objective of the mechanism is to provide an efficient
     energy allocation by matching the buy and sell orders of the participants appropriately.
     Finally, this should happen in near real-time granularity.    
    
    \item[Pricing Mechanism (C5):] The market mechanism (C4) includes the pricing mechanism 
     and supports the efficient allocation of energy supply and demand. 
     The traditional energy price is composed of large parts of taxes and surcharges.
     On the contrary, in LEM different fees come to bear, for example in case of a 
     physical microgrid. Hence, RES typically have almost zero marginal cost, prosumer can 
     price their energy above all appropriate taxes and fees to make profit. 
    Thus, the energy price should be linked to the availability of energy. In other words,
    a surplus of energy should lower the LEM energy price while a lack of energy increases the 
    market price. From an economic point of view, LEM are beneficial to their 
    participants as long as the average energy price is lower than the external grid price.
        
    \item[Energy Management Trading System (C6):] The task and goal of the \textit{energy management trading system (EMTS)}
     is to automatically ensure energy supply for a respective market participant.
     Therefore, the EMTS needs access to the energy-related data of the participant, like 
     the real-time demand and supply. The EMTS uses this data to forecasts consumption and
     generation and develops a bidding strategy accordingly. 
     Moreover, the EMTS trades the predicted amounts on the provided market platform 
     and aims to maximize the revenue and minimize the energy costs. 
     For this reason, the EMTS needs to have access to the market participant’s
     blockchain account to be capable to automatically perform energy transactions.

    \item[Regulation (C7):] It needs to be determined how LEM 
     fit into the current energy policy and which market design is allowed, how 
     taxes and fees are distributed and billed. Likewise, it needs to 
     be determined in which way the local market is integrated into the traditional
     energy market and energy supply system.
     All these emerging issues are specified by the legislative regulation.
    
\end{description}


\clearpage