\section{Local Energy Markets}
To begin with, we revive the stated problem of a successful integration of the increasing amount of
distributed renewable energy sources (RES), as already adressed 
in section \ref{sec:research_motivation}. Due to the centralized generation by large power plants
and the current design of the wholesale markets, the existing grid is not suitable designed
to react in real-time to a significant growing 
amount of distributed RES \shortcite{mengelkamp2018designing} \shortcite{ampatzis2014local}.

In order to successfully integrate and use these RES, new approaches are necessary \shortcite{mengelkamp2018blockchain}.
A possible solution to those technical and market problems is Peer-to-Peer (P2P) energy 
trading in local energy markets (LEM) \shortcite{long2017feasibility}. 

This section will explain the general concept of local energy markets 
and present market components for building efficient LEM. 

In the traditional centralized energy system, large power plants that operate according to a
centralized coordination mechanism, supply a large amount of customer with energy, which are located 
within a wide area (for example a country or a state) \shortcite{mengelkamp2018designing}.

In contrast, decentralized energy systems consists of small-scale energy generators that are 
only used by a small number of people and located close to the energy consumption point \shortcite{mengelkamp2018designing}.
Those local energy markets provide a market platform, market mechanism and market access
for small-scale prosumer and consumer, to trade locally generated energy within their community.
In this case, a community presents a group of geographically and socially close energy agents.
Moreover, all participants buy or sell energy directly with each other by using the provided market platform
without intermediation by conventional energy suppliers \shortcite{zhang2017review}.

In addition, if prosumer have a surplus in electricity, they have the opportunity 
to curtail it, store it in a energy storage device or export it back to the main power grid \shortcite{zhang2017review}.
Considering, in the traditional centralized energy system, trading of energy is mainly unidirectional.
As stated above, the electricity is usually transmitted from large-scale power plants to 
consumers over long distances, while the cash-flow goes the opposite way. 
On the contrary, the P2P energy trading in a LEM encourages multidirectional trading within 
a local geographical community \shortcite{zhang2017review}.

Due to this, LEM promote the consumption of energy close to its generation and, therefore, 
support sustainability and the efficient use of local resources \shortcite{mengelkamp2018designing}.
Furthermore, local energy markets enable (near) real-time pricing and facilitate a local balance
of supply and demand \shortcite{mengelkamp2018blockchain}. 

Further, as stated by \shortciteA{mengelkamp2018designing},
the participants of a LEM do not necessarily have to be physically connected. A virtual microgrid
describes the aggregated control of multiple energy producers, prosumers and consumers in a virtual 
community. Further, the revenue potential can be increased significantly by expanding a physical 
microgrid to inlude virtual participants. 

Referring to \shortciteA{jimeno2011architecture}, a LEM has a specific characteristic
that distinguish it from other aggregation of DER systems. A microgrid has the opportunity
to operate either connected to the main grid or islanded from it. In other words,
this allows a LEM to run disconnected from the main grid, in case it fails or 
the power quality is not satisfactory. Thus, participants of a LEM have a higher quality 
of supply for the loads within it. Additionally, it offers a way of obtaining cheaper 
and cleaner energy for all participants, if elements operated in a LEM taking into 
account by economic and emission policies.

Moreover, P2P energy trading in LEM requires advanced communication and data 
exchanges between the different parties, which makes central management and 
operation more and more challenging. The implementation of LEM needs 
local distributed control and management techniques. \shortcite{andoni2019blockchain}. 
\shortciteA{zhang2017review} stated that, P2P energy trading is often
enabled by ICT-based online services. Moreover, \shortciteA{mengelkamp2018designing} explain that 
the new and innovative blockchain technology as an emerging ICT, 
offers new opportunities for decentralized market designs.
It is designed to enable distributed transactions without 
a central trusted entity.
Accordingly, blockchain can help addressing the challenges faced by 
decentralized energy systems. 

However, blockchain is not a matured technology 
yet and there are several barriers in using them, especially 
for researcher who do not have a technically background. Therefore, this research 
will use a blockchain in the LEM simulation platform as the underlying ICT and will give a
detailed introduction to the blockchain technology in section \ref{sec:about_blockchain}.
In conclusion, the developed simulation platform removes existing technical barriers 
for practicioners and enables the usage of blockchain technology for those.

\subsection{Components of local energy markets}
\shortciteA{mengelkamp2018designing} developed in his research seven components for a efficient
operation of blockchain-based local energy markets. This subsection will name and briefly 
illustrate each component. Further, the compliance of the developed open blockchain-based
LEM simulation and the seven components will be examined.

\begin{description}
    \item[Microgrid setup (C1):] In general, an explicit objective, a definition of the market 
     participants and the form of the traded energy must be well defined. 
     A LEM can have different, often contradictory objectives. Especially in the
     design of the market mechanism, the implementation of the objective plays an important role.
     Next, a significant number of market participants is needed, who trade energy among each other.
     Moreover, a part of the market participants needs to be able to produce energy. 
     Finally, the form of the traded energy must be described, for example electricity, heat or a 
     combination of them. Additionally, the way of energy transportation must be specified.
     Will be the traditional energy grid used or a physical microgrid build. 
    
    \item[Grid connection (C2:)]
    
    One or several connection points towards the superordinate
    grid are a key component [31] and must be well defined for balancing
    energy generation and demand within the microgrid with the help 
    of the superordinate grid. At these points, energy flows towards 
    the respective superordinate grid can be metered to accurately measure 
    the microgrid’s performance. A distinctive difference exists between a 
    physical microgrid, which consists of an actual power distribution microgrid, 
    and a virtual microgrid, which simply links the microgrid participants 
    over an information system (C3). Contrary to a physical microgrid, 
    virtual microgrids cannot physically decouple from the superordinate grid. 
    Physical microgrids typically have a limited number of connection points to
    ensure an efficient grid connection but also to swiftly decouple from the grid in case of power outages. For extended decoupling and island-mode operation, micro-
    grids need a large amount of their own energy generation capacity and 
    flexibility to ensure an appropriate level of supply 
    security and resiliency. Flexibility can be provided in 
    form of demand or gener- ation flexibility and storage capacities [47]. 
    As energy is a physical good and transmitted on constrained grids, energy 
    flow problems, e.g. grid congestion [48], need to be taken into account.
    
    \item[Information system (C3):]
    
    
    \item[Market mechanism (C4):] 
    
    
    \item[Pricing mechanism (C5):]
    
    
    \item[Energy management trading system (C6):] 
    
    
    \item[Regulation (C7):] 
\end{description}

\begin{comment}
    # Mengelkamp \shortcite{mengelkamp2018designing}
    
    An exemplary microgrid energy market scenario of residential consumers and 
    prosumers (consumers with photovoltaic (PV) systems) is shown in Fig. 2. 

\end{comment}

\clearpage