To begin with, this chapter will give an overview of the general purpose, the contained components and the fundamental functionality of a blockchain. 

\subsection{General Purpose}
In general, a blockchain can be described as a digital data structure that can be understand as a shared and distributed database, containing a continuous expanding and chronological log of transactions \cite{andoni2019blockchain}. Besides, various types like digital transactions, data records and executables can be stored in this digital data structure. The data transmission in a blockchain is comparable with copying data from one computer to another. However, the resulting challenge is that the system needs to ensure that the data is copied just once \cite{andoni2019blockchain}. For example, in the domain of cryptocurrencies, this is equal to sending a coin from one wallet to another. In this case, the system needs to validate that this coin is spended just once and there is no double-spending. A conventional solution for this problem is a third intermediary. To come back to the stated example, the third intermediary is represented by a traditional bank, which store, protect and continuously update the valid state of the ledger \cite{andoni2019blockchain}. But, in some cases central management is not practicable or reasonable. Reasons for this are possible intermediary costs or a high degree of trust of the users into the intermediary who operates the system. Further, central management has a significant disadvantage because of a single point of failure. Hence, the centralized system is fragile to technical problems as well to external malicious attacks \cite{andoni2019blockchain}.
Consequently, the main reason of bockchain technologies is the removal of such third trusted intermediaries through a distributed network of various users, who cooperating together to verify transactions and protect the validity of the ledger

\subsection{Architecture}
This subsection covers the architectural design of a blockchain and present all contained components in detail. To begin with, a blockchain is a sequence of blocks, which holds a complete list of transaction records.


Blockchain is a sequence of blocks, which holds a complete
list of transaction records like conventional public ledger [14]. Figure 1 illustrates an example of a blockchain. With a previous block hash contained in the block header, a block has only one parent block. It is worth noting that uncle blocks (children of the block’s ancestors) hashes would also be stored in ethereum blockchain [15]. The first block of a blockchain is called genesis block which has no parent block. We then explain the internals of blockchain in details \cite{{}



\subsection{Consensus Algorithms}

\subsection{Ethereum Blockchain}

\subsubsection{Smart Contracts}