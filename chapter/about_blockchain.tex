\section{Blockchain}
\label{sec:about_blockchain}

We stated that the implementation of LEM
needs local distributed control and management techniques and emphasize
that the blockchain technology as an emerging ICT offers new opportunities for decentralized
market designs and helps to address the challenges faced by decentralized energy systems. 
Therefore, this research applies a blockchain in the BLEMS as the underlying ICT, which is why
this chapter gives an overview of the general purpose, 
the contained components and the fundamental functionality of a blockchain. 

\subsection{General Purpose}
In general, a blockchain is a digital data structure that can be understood as a shared and 
distributed database, containing a continuously expanding and chronological log of transactions \shortcite{andoni2019blockchain}. 
Various data types like digital transactions, data records and executables can be stored in this digital data structure. 
The data transmission in a blockchain is comparable with copying data from one computer to another. However, the resulting challenge 
is that the system needs to ensure that the data is copied just once \shortcite{andoni2019blockchain}. 
For example, in the domain of cryptocurrencies, this is equal to sending a coin from one wallet to another. 
The system needs to validate that this coin is spent just once and there is no double-spending. 
A conventional solution for this problem is a third intermediary. To give a tangible example, the third
intermediary can be described by a traditional bank, which stores, protects and continuously updates the valid state of 
the ledger \shortcite{andoni2019blockchain}. In some cases, central management is not practicable or reasonable due to 
possible intermediary costs or a high degree of trust of the users into the intermediary who operates 
the system. Central management has a significant disadvantage because of a single point of failure. 
Hence, the centralized system is fragile to technical problems as well as to external malicious attacks \shortcite{andoni2019blockchain}.
Consequently, the main reason for blockchain technologies is the removal of such third 
trusted intermediaries through a distributed network of various users, who cooperating 
to verify transactions and protect the validity of the ledger.

\subsection{Components}
This subsection covers the architectural design of a blockchain and presents the 
contained components in detail. Due to the plurality of the blockchain technologies, 
each of the technology slightly differs in design and components. 
The following explanations are oriented towards the Ethereum, as it used 
in the implementation of the BLEMS.

\subsubsection{World State}
\label{sec:world_state}
Referring to \shortciteA{wood2014ethereum}, Ethereum 
can be seen as a transaction-based state machine.
As you can see in Figure \ref{figure:state_transition}, the state machine starts with a so-called 
\textit{genesis state}. This is analogous to a blank sheet. 
In this state, no transactions have happened on the network. 
Next, transactions are executed incrementally and the state of the Ethereum world 
changes into new states and morph it into some final state. 
Consequently, the final state 
is accepted as the canonical version of the world of Ethereum and 
represents at any time the current state.

\begin{figure}[htbp]
	\centering
	\includegraphics[width=.75\linewidth]{./figures/state_transition.pdf}
	\caption{World State: Transition of States}
	\label{figure:state_transition}
\end{figure}

The world state arises out of a mapping of a key-value pair for every account which exists 
on the Ethereum network \shortcite{wood2014ethereum}. 
The key constitutes the address of an Ethereum account and the value presents 
the account's state, which contains detailed information. 
However, the world state is not stored on the blockchain itself. 
This mapping is stored and maintained in a modified data structure 
called a \textit{Patricia Merkle tree}. This tree is stored off-chain 
in a simple database backend (i.e. on a computer running an Ethereum client), 
also known as the \textit{state database} in the Ethereum world \shortcite{wood2014ethereum}. 
To get a better understanding of the operating principles of the blockchain, 
it is necessary to get an idea of how a Patricia Merkle tree works. 
A Patricia Merkle tree is a type of binary tree, which consists 
of a set of nodes. It has a large amount of 
leaf nodes, containing the underlying data. 
Further, a set of intermediate nodes, where each node is 
the hash of its two children, and finally, one single root node, 
representing the top of the tree that is also built out of its 
two child nodes \shortcite{buterin2013next} \shortcite{wood2014ethereum}.
As mentioned before, the leaf nodes contain the stored data 
by splitting these data into \textit{chunks}. Afterwards, these \textit{chunks} are split into \textit{buckets}. 
Then, each bucket gets hashed and the same process repeats, traversing 
upwards the tree until the total number of hashes remaining becomes 
only one and the root node is reached \shortcite{ethereum_blog}. 

\begin{figure}[htbp]
	\centering
	\includegraphics[width=.8\linewidth]{./figures/merkle_tree.pdf}
	\caption{Example of a Patricia Merkle tree}
	\label{figure:merkle_tree}
\end{figure}

Therefore, any change to the underlying data, stored in a \textit{leaf node}, causes a change
of the hash of the node. Each parent's node hash depends on the data of its children. 
Due to this, any change to the data of a child node causes the parent hash to change. 
This procedure repeats traversing upwards until the root node. 
Hence, any change to the data at the leaf nodes affects the root hash \shortcite{ethereum_wiki}. 

\begin{figure}[htbp]
	\centering
	\includegraphics[width=.8\linewidth]{./figures/merkle_tree_change.pdf}
	\caption{Example of a data change in a leaf node}
	\label{figure:merkle_tree_change}
\end{figure}

Because of this characteristic, it is not necessary to compare the data of the entire tree. 
It is sufficient to compare the single root hash to ensure that all the data are the same. 
This property is very important because it makes it possible to store only the hash of 
the root node to represent a state of the Ethereum world. 

\subsubsection{Block} 
\label{sec:block}
As described in section \ref{sec:world_state}, whenever transactions are executed, 
the state of the Ethereum world changes into a new state. 
Every state and the belonging transactions, which transform the prior state into 
the new state, is cumulated in a so-called block. 
That means the history of the Ethereum world is a linkage of states which are represented by blocks.
That is where the name blockchain comes from. A blockchain is a sequence 
of blocks, which holds a complete list of transaction records \shortcite{zheng2017overview}. 
Moreover, a block is a collection of different relevant information and 
consists of the \textit{block header} and the \textit{block body}, 
which contains the list of transactions. Following \shortciteA{wood2014ethereum}, 
the subsequent pieces of information are contained in the block header:

\begin{description}
	\item[Parent Hash:] Hash of the parents' block's header. Therefore, every block points to his ancestor. Due to this contained attribute, a chain arises out of the single blocks.
	\item[Beneficiary:] Miners address to which all block rewards from the successful mining of a block are transferred.
	\item[State Root:] Hash of the root node of the state tree after a block and its transactions are finalized. As mentioned in section 
	\ref{sec:world_state}, the state tree is the unambiguous global state in the Ethereum world. It is used as a secure unique identifier for the state and the state root node is cryptographically dependent on all internal state tree data.
	\item[Transactions Root:] Hash of the root node of the transaction tree. This tree contains all transactions in the block body. In contrast to the state tree, there is a separate transaction tree for every block. 
	\item[Receipts Root:] Every time a transaction is executed, Ethereum generates a transaction receipt that contains information about the transaction execution. This field is the hash of the root node of the transactions receipt tree and as the transaction tree, there is a separate receipt tree for every block.
	\item[Difficulty:] Measure of how hard it was to mine this block – a quantity calculated from the previous block’s difficulty and its timestamp
	\item[Number:] Quantity equal to the number of blocks that precede the current block in the blockchain.
	\item[Gas Limit:] Quantity equal to the current maximum gas expenditure per block. Each transaction consumes gas. The gas limit specifies the maximum gas that can be used by the transactions included in the block. It is a way to limit the number of transactions in a block.
	\item[Gas Used:] Quantity equal to the total gas used in transactions in this block.
	\item[Timestamp:] Record of Unix’s time at this block’s inception.
	\item[Nonce:] Hash that verifies a sufficient amount of computation has been done on this block. Further, it is a number added to a hashed block that, when rehashed, meets the difficulty level restrictions. The nonce is the number that blockchain miners are solving for.
\end{description}

We introduced Ethereum as a transaction-based state machine, transforming 
one state into another through the execution of transactions. Further, we explained 
how these individual states are stored and emphasizes the meaning of this storage. 
We pointed out that an Ethereum state is represented by blocks and described 
all relevant information contained in it.
However, we didn't introduce how many transactions be part 
of a block and who build and validate a block. Therefore, 
we introduce the transaction object and its life cycle in-depth.


\subsubsection{Transaction}
\label{sec:transaction}
A transaction is a basic method for Ethereum accounts to interact with each other. 
In the Ethereum world exists two different types of transactions. 
Those, which result in a so called \textit{message call}, which can be seen as a 
traditional transaction, and those which result in a \textit{contract creation} \shortcite{wood2014ethereum}. 
Both different types of transaction share the following common attributes \shortcite{wood2014ethereum}:

\begin{description}
	\item[Nonce: ] This attribute is discontiguous of the block attribute. In contrast to the block attribute, the nonce of a transaction keeps track of the total number of transactions that an account has executed. 
	\item[Gas Price: ] As mentioned in section \ref{sec:block}, each transaction consume gas. Gas can be seen as fees. This attribute presents the price per unit of gas for a transaction\footnote{A website to see current gas prices: https://ethgasstation.info/index.php}. 
	\item[Gas Limit: ] This attribute is similar to the block attribute of the same name. In this case, it is a quantity equal to the current maximum gas expenditure per transaction.  
	\item[To: ] In case of a message call, this attribute contains the address of the recipient. Otherwise, in case of a contract creation this value remains empty. 
	\item[Value: ] In case of a message call, this attribute contains the amount of Wei, which will be transferred to the recipient. 
	In case of a contract creation, this value contains the amount of Wei as an initial endowment of the contract. 
	Wei is a subdivision of the cryptographic currency of Ethereum and 1 Wei is defined as $10^{-18}$ Ether.
\end{description}

\subsubsection{Transaction Life Cycle}
\label{sec:transaction_lifecycle}
This subsection outlines the life cycle of a transaction.
We take a simple message call as an example 
and pass through the entire flow of how this transaction 
gets executed and permanently stored on the blockchain. 
During this process, many basic concepts of a blockchain are introduced. 

To begin with, someone wants to send an arbitrary amount of ether to someone else. 
Consequently, a transaction with the respective attributes, which are stated 
in section \ref{sec:transaction}, will be created. As a first step, this transaction 
will be signed. That means the one who wants to execute the transaction needs to prove the 
ownership of the account. Otherwise, someone else could execute a transaction on your behalf. 
The way to prove this is by signing the transaction with the corresponding private
key of this account \shortcite{transaction_life_cycle}. 
Next, the signed transaction is submitted to the related Ethereum node. 
Then, the node will validate the signed transaction. 
After successful validation, the node will send the transaction to 
its peer nodes, who again send it to their 
peer nodes and so on \shortcite{transaction_life_cycle}. 
As soon as the transaction is broadcast to the network, 
the related node will declare a transaction ID, which is the hash 
of the signed transaction and can be used to track the 
status of the transaction\footnote{A Website to track transactions: https://etherscan.io/}. 

\begin{figure}[htbp]
	\centering
	\includegraphics[width=1\linewidth]{./figures/node_network.pdf}
	\caption{Node network demonstrating a transaction broadcast}
	\label{figure:node_network}
\end{figure}

\clearpage

Figure \ref{figure:node_network} describes an Ethereum network 
which contains a mix of miner nodes and non-miner nodes. 
The non-miner nodes can be distinguished into different types of nodes with different 
behavior, which are explained in detail in section \ref{sec:nodes}. 
The miner nodes are the ones who processing transactions into blocks \shortcite{transaction_life_cycle}. 

Mining nodes contain and maintain a pool of transactions where they collect incoming transactions. 
They sort the transactions by gas price. 
However, there are no specific rules on how the nodes sort the transactions. 
A common configuration is to sort the transaction by gas price in descending order to 
optimize for a higher pay \shortcite{transaction_life_cycle}.

\begin{longtable}{c|c}
	\caption{Transaction Pool of Mining Node} 
	\label{table:sorted_gas_prices}
	\\
	\textbf{Signed Transaction} & \textbf{Gas Price} \\
	\hline
	hashOfTransaction1 & 15 Gwei \\
	hashOfTransaction2 & 13 Gwei \\
	hashOfTransaction3 & 11 Gwei \\
	hashOfTransaction4 & 10 Gwei \\
	hashOfTransaction5 & 7 Gwei \\
\end{longtable} 

The mining node takes transactions from the pool and processes them into a pending block. 
A block can only contain a certain number of transactions due to the block 
attribute \textit{gas limit}, mentioned in section \ref{sec:block}. 
That means the mining node can only process so many transactions into one 
block until the gas limit is reached\footnote{A website to see 
current attributes: https://ethstats.net/}. For example, 
imagine that the current gas limit of a block is 28 Gwei. 
Next, we look at the transaction pool of a mining node in 
table \ref{table:sorted_gas_prices}. Two blocks are 
required to process all transactions from the pool. The first two transactions 
are contained in the first block and the second block contains the remaining 
three transactions. 
So far, it is described how a pending block will be created and the transaction 
limit of a block is explained. In the next section \ref{sec:mining}, 
the process of a pending block to a valid block is outlined in depth. 
After the validation process of a block is finished successfully, 
the life cycle of a transaction is terminated. In other words, the transaction is anchored 
in the blockchain and modified the world state as mentioned in section \ref{sec:world_state}.

\subsubsection{Mining}
\label{sec:mining}
In this section, the steps of the validation process of a pending block is described. 
We explain which node builds a new block and how this regenerated block is accepted by the rest of the network.
The process responsible for this is called \textit{consensus mechanism}.
There are many various consensus algorithms, each of them provides different features,
advantages, and disadvantages. 

To a large extent, the key performance drivers of a blockchain
like transaction speed, scalability and security depend on the embedded consensus algorithm \shortcite{andoni2019blockchain}. 

In this section, we expound the \textit{Proof of Work (PoW)} mechanism, 
which is at the time of writing the used consensus mechanism in the Ethereum blockchain.
PoW enables high scalability in terms of the number 
of nodes and clients \shortcite{vukolic2015quest}, whereas it provides a poor transaction speed and high 
power consumption due to the solving of a cryptographical puzzle, which requires 
significant computational effort \shortcite{andoni2019blockchain}. 

% erklärung crypthograpgical puzzle
PoW is a random process that is not predictable and therefore only solvable through a 
trial and error approach \shortcite{bitcoin_wiki_pow}. 
All mining nodes compete with each other and the goal is to achieve a hash output that 
is lower than a given specified target \shortcite{andoni2019blockchain}.
The hash output includes and comprises out of the \textit{Parent Hash}, \textit{Transaction Root} and
\textit{Nonce}, which are all part of the block headers data, mentioned in \ref{sec:block}.
However, the Parent Hash and Transaction Root are given and immutable, thus the Nonce
is the value that all the mining nodes are solving for\footnote{A interactive blockchain demo: https://anders.com/blockchain/}.
Consequently, the mining nodes modify the Nonce until the hash output is lower than the required target \shortcite{andoni2019blockchain}.
The following pseudocode in Listing \ref{lst:pow_mechanism} illustrates the PoW procedure:

\vspace{7mm}
\begin{lstlisting}[label={lst:pow_mechanism}, caption={Pseudocode of PoW Mechanism}]
	targetValue = 00005cdf6d384113165841052dfd4638eaf756ac
	parentHash = abbecf2d59eacbde676c3f1f0bfcf124f8c68209
	transactionRoot = 917c631e5ee59596859910759c8ed76a21252010
	nonce = 1
	valid = False

	while(not valid):
			hashOutput = sha256(parentHash+transactionRoot+nonce)
		
			if(hashOutput <= targetValue):
					valid = True
			else:
					nonce += 1

\end{lstlisting}
\clearpage

Finally, when one mining node successfully calculated the Nonce and therefore reaches the target value, 
it broadcasts the block to all other mining nodes in the network. Now, all other nodes mutually validate the correctness of the hash value
by recalculating the hash output and verifying if it is lower than the target value. 
If the block is validated through all other mining nodes, all nodes will append this new block to their own blockchains \shortcite{zheng2017overview}.  

% erklärung multiple chains
Considering, the network is decentralized and the simultaneous generation of valid blocks is possible when multiple nodes find 
the suitable Nonce at a nearly same time \shortcite{zheng2016blockchain}. In this case, branches will be generated as shown in Figure \ref{figure:blockchain_branches}

\begin{figure}[htbp]
	\centering
	\includegraphics[width=.9\linewidth]{./figures/blockchain_branches.pdf}
	\caption{Scenario of Blockchain Branches}
	\label{figure:blockchain_branches}
\end{figure}

Nevertheless, it is unlikely that two competing branches will generate the next block simultaneously again. 
In the PoW protocol, the longer branch is judged as the authentic one. 
In Figure \ref{figure:blockchain_branches}, two branches arised out of 
two simultaneous valided blocks Block 4a and Block 4b. 
After that, all mining nodes accept both branches and start working on 
the next consecutive block until a longer branch is found \shortcite{andoni2019blockchain}. 
As you can see in Figure \ref{figure:blockchain_branches}, Block 4b and Block 5 forms the longer 
chain, therefore all miners on the branch with Block 4a will switch to 
the longer branch \shortcite{zheng2017overview}. 

To sum up, this section joins and completes the previous section \ref{sec:transaction_lifecycle}.
The consensus mechanism PoW is described and outlined in pseudocode, whereby
the validation process of newly generated blocks is explained and illustrated.
\clearpage

\subsubsection{Ethereum Clients}
\label{sec:nodes}

The expressions Ethereum client and Ethereum node can be used interchangeably. 
As mentioned earlier in the previous section \ref{sec:transaction_lifecycle} 
and illustrated in the Figure \ref{figure:node_network}, clients can be distinguished into different types. 
This subsection describes the concept of an Ethereum client in general and gives a brief overview of the different
types.

In general, an Ethereum client is a software application 
that implements the Ethereum specification, which is specified in the Ethereum yellow paper \shortcite{wood2014ethereum}.
A client communicates over the P2P network with other Ethereum clients.
Although these clients can be implemented in different programming languages, they all communicate through
the standardized Ethereum protocol, whereby they can operate and interact with the same Ethereum network \shortcite{antonopoulos2018mastering}. 

We present a list of the current main implementations of the Ethereum protocol,
since the Ethereum blockchain is not officially implemented in a particular programming language:

\begin{itemize}
	\item \texttt{Parity}, written in Rust
	\item \texttt{Geth}, written in Go
	\item \texttt{cpp-Ethereum}, written in C++
	\item \texttt{pyethereum}, written in Python
	\item \texttt{Harmony}, written in Java 
\end{itemize}

While all clients differ from each other, they share some fundamental features \shortcite{ethereum_clients}.

First, each client can join the P2P Ethereum network and they all synchronizing a local copy of the blockchain.
There are a few modes of synchronizing, which comes with miscellaneous advantages and disadvantages, 
which makes a significant difference in the health and resilience of the decentralized network.
However, each client is capable of broadcasting new transactions to the network and creating and 
managing accounts \shortcite{ethereum_clients}.
In general, the clients can be distinguished into two different types, the so-called \textit{full node} and 
\textit{light node}. 

\paragraph{Full Node:} A full node downloads all history data P2P from
another full node. This requires a significant amount of hardware and bandwidth resources \shortcite{antonopoulos2018mastering}.
Afterwards, it simulates every transaction in the ledger
and execute the whole deployed source code to recalculate the state of each existing block \shortcite{ethereum_clients}.
Therefore, the amount of independently operating and geographically dispersed full nodes 
is a crucial and very important indicator for the health, resilience and 
censorship resistance of a blockchain \shortcite{antonopoulos2018mastering}. 
However, a full node is not necessarily a mining node. It authoritatively validates all transactions, but 
to participate in the mining competition, an additional software called \textit{ethminer}
 is needed\footnote{GPU Mining Worker: https://github.com/ethereum-mining/ethminer}

\paragraph{Light Node:} To begin with, a light node is not a separate piece of software from the clients that 
run a full node. The difference lies in the light mode for synchronization. 
In contrast to a full node, a light node only synchronize the block headers and the 
current state of the chain \shortcite{ethereum_light_node}.
Therefore, a light node cannot run transactions throughout the history of the 
whole blockchain. For this reason, it does not contribute to the health and resilience of 
the network like a full node and traditionally cannot act as a mining node \shortcite{ethereum_clients}. 
In addition, a light node has some further limitations. For instance, they are not 
capable to monitor pending transactions from the network and the hash of a transaction
is not sufficient to locate the transaction. Moreover, to perform certrain types of operations,
a light node relies on requests to full nodes, whereby they can be far slower to query the chain \shortcite{ethereum_light_node}.


\subsubsection{Smart Contract}
\label{sec:smart_contract}
This section deals with smart contracts in the Ethereum world. It presents a definition 
of smart contracts and explains its characteristics.

The term smart contract is a bit misleading, since the Ethereum smart contracts 
are neither smart nor legal contracts.
An Ethereum smart contract is an immutable computer program that runs deterministically in the context of an
Ethereum Virtual Machine as part of the network protocol \shortciteA{antonopoulos2018mastering}.
Immutability means that once the contract is deployed into the blockchain, it is not possible to make any
changes in the source code. The only way to modify the contract is to deploy a whole new instance. 
Deterministic means that the result of the execution is the same for all who run it, 
depending on the context of the transaction that initiated its execution and the state of the blockchain
at the moment of execution \shortciteA{antonopoulos2018mastering}.
As already mentioned in section \ref{sec:transaction}, in the Ethereum world exists two different types of transactions.
Those which result in a message call, and those which result in contract creation. 
That means smart contracts are deployed through special contract creation transactions into the blockchain. 
The special thing about these transactions is that the recipient of the transaction remains empty, whereby the transaction
will be sent to a special destination address called \textit{zero address (0x0)}. Each contract is identified and reachable via
a common Ethereum address, which can be used in a transaction as the recipient to send funds to the contract or to call 
one of the contract functions. Regardless, in contrast to an externally owned account, there are no keys associated with 
an account of a smart contract \shortciteA{antonopoulos2018mastering}. For this reason, the creator of a smart contract does not have any special rights at the 
protocol level. Although, it is possible to get those special rights if you explicitly 
specify them in the source code of the contract.
Furthermore, contracts are only active and run if they called by a transaction. 
A contract can call another contract and so on, but, 
the first call has always come from an externally owned account. 
A contract will never run on his own or in the background \shortciteA{antonopoulos2018mastering}. Additionally, contracts are not executed in parallel.
They always execute consecutively, hence the Ethereum blockchain can be considered as a single-threaded machine. 
It should be noted that transactions are atomic regardless of how many contracts they call. Transactions execute in
their entirety and any changes in the global state are only recorded if all executions terminate successfully. 
If the executions fail, all the changes in state are reverted as if the transaction never ran. Nevertheless, a failed
execution of a transaction is recorded as having been attempted and the gas fees spent for the execution are withdrawn
from the originating account.

\clearpage