\section{The Bundle-Trading-Market Framework}
As already briefly introduced in the sections \ref{sec:research_motivation} 
and \ref{sec:Distributed Resource Optimization}, this chapter covers and explain in detail
the developed market-based optimization algorithm, which is also called the 
\textit{bundle trading market framework} or short \textit{BTM}. 
In the proposed framework by \shortciteA{guo2012computational}, independent, self-interested
agents trading bundled resources in a double-auction market, which run by a dealer. 
In this case, the dealer replaces the central authority and agents represent distributed
entities. Generally, the \textit{BTM} consists of a master problem that solves 
the market-matching problem, which is managed by the dealer, 
and  subproblems that solve the bundle-determination problem of the dealer.
Further, the solution process is an iteration between the market-matching problem and the 
the bundle-determination problems \shortcite{guo2007market}.
Considering, the \textit{BTM} constitutes a market-based decomposition method for decomposable linear systems,
which can be easily implemented to support real-time optimization of distributed systems \shortcite{guo2007market}.
Therefore, the \textit{BTM} is a suitable market mechanism for the concept of a LEM and will be used
as such in the developed blockchain-based LEM simulation platform. 


\begin{comment}
### RESEARCH MOTIVATION - BTM FRAMEWORK
Further, Guo, Koehler and Whinston developed a market-based optimization algorithm, 
which solves a distributed system optimization problem by self-interested agents 
iteratively trading bundled resources in a double auction market run by a dealer. 
The authors called it bundle trading market framework or short BTM \shortcite{guo2007market}. 
The central problem of the stated market-based optimization algorithm can interpreted 
as the welfare optimization of all participants in a LEM. The dealer, 
which runs the double auction market, maximizes the welfare through allowing 
agents to trade their preferable bundles of energy. Hence, the stated 
BTM implemented on basis of a blockchain as underlying ICT can depict the concept of a LEM.

### RESEARCH MOTIVATION - BRINGING ALL TOGEHTER
Consequently, this research will bring all the introduced approaches together and 
develop an open blockchain-based LEM simulation, which enables the research approach 
based on the three stated elements of CB. The platform will be realized through the 
introduced optimization algorithm with a blockchain as the underlying ICT. 
That means, the smart contract takes the role of the market dealer and the 
self-interested agents represent the individual participants in a LEM. 
The focus of this paper will be on the implementation and software design of 
the open blockchain-based LEM simulation platform.

### LITERATURE REVIEW
To begin with, Fan et al. outline a new approach for the development of an 
information system which can be used for the problem of a supply chain. 
The concept demonstrate a decentralized decision making process that is 
realized through the design of a market-based coordination system which 
incite the pariticipants to act in a way that is beneficial to the overall 
systems \shortcite{fan2003decentralized}. Further, Guo et al. revive this 
concept and develop a market-based decomposition method for 
decomposable linear systems, that can be easily implemented to 
support real-time optimization of dsitributed systems \shortcite{guo2007market}. 
They prove that the system optimality can be achieved under a dynamic 
market-trading algorithm in a finite number of trades. 
Moreover, the outlined alogrithm can be operated in synchronous 
and as well in asynchronous environments. 
Later on, Guo et al. extend their stated concept to a dynamic, asynchronous 
internet market environment \shortcite{guo2012computational}. 
Additionally, they examine how various market design factors 
like dealer inventory policies, market communication patterns, 
and agent learning strategies affect the computational market efficiency 
and implementation. Finally, also this time, Guo et al. prove finite 
convergence to an optimal solution under all these different schemes. 

\end{comment}
