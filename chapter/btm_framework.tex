\section{The Bundle-Trading-Market Framework}
To start with, we explained in section \ref{sec:lem} that local energy markets enable the trading 
of locally generated energy through a market platform, market mechanism and market access for 
small-scale prosumer and consumer. 
In addition, we presented in section \ref{sec:components_of_local_energy_markets} 
a market mechanism as one of seven core components for a efficient operation of blockchain-based
local energy markets.

On account of that, this chapter covers and explain in detail the applied market mechanism in the 
developed open blockchain-based LEM platform. 

As already briefly introduced in the sections \ref{sec:research_motivation} 
and \ref{sec:Distributed Resource Optimization}, the applied market mechanism is implemented 
through the developed market-based optimization algorithm, which is also called the 
\textit{bundle trading market framework} or short \textit{BTM}. 
In the proposed framework by \shortciteA{guo2012computational}, independent, self-interested
agents trading bundled resources in a double-auction market, which run by a dealer. 
In this case, the dealer replaces the central authority and agents represent distributed
entities. Generally, the \textit{BTM} consists of a master problem that solves 
the market-matching problem, which is managed by the dealer, 
and  subproblems that solve the bundle-determination problem of the dealer.
Further, the solution process is an iteration between the market-matching problem and the 
the bundle-determination problems \shortcite{guo2007market}.
Considering, the \textit{BTM} constitutes a market-based decomposition method 
for decomposable linear systems,
which can be easily implemented to support real-time optimization 
of distributed systems \shortcite{guo2007market}.
Likewise, the central problem of the stated market-based optimization algorithm can 
interpreted as the welfare optimization of all participants in a LEM. 
The dealer, which runs the double auction market, maximizes the welfare through allowing 
agents to trade their preferable bundles of energy.
Therefore, the \textit{BTM} is a suitable market mechanism 
for the concept of a LEM and will be used as such in the developed blockchain-based LEM 
simulation platform. 

\subsection{Problem Overview}
This subsection will give an introductory description to the overall problem. 
With reference to \shortciteA{guo2012computational}, we consider a distributed 
system with $k$ independent agents.
In addition, we have a central problem and individual 
problems of the $k$ individual agents, which both can be expressed as the following 
linear programs. Besides, the \textit{BTM} presented by \shortciteA{guo2007market} requires 
a nondegenerate central problem with a bounded solution. In contrast to discrete markets where 
only integer number of units can be traded, the proposed framework facilitate continuous 
trade amounts. 

\paragraph*{Central problem}
\begin{equation}
    \begin{array}{ll@{}ll}
        \underset{x_{j} \geq 0}{\text{min}}  & \displaystyle\sum\limits_{j=1}^{k} d^{T}_{j}x_{j} &\\
        \text{s.t.}& \displaystyle N_{j}x_{j} \geq n_{j}, &&i=1 ,..., k  \\
                    & \displaystyle\sum\limits_{j=1}^{k} C_{j}x_{j} \leq c
    \end{array}
\end{equation}

\paragraph*{Agent problem (j=1, ..., k)}
\begin{equation}
    \begin{array}{ll@{}ll}
        \underset{x_{j} \geq 0}{\text{min}}  & \displaystyle\sum\limits_{j=1}^{k} d^{T}_{j}x_{j} &\\
        \text{s.t.}& \displaystyle N_{j}x_{j} \geq n_{j}, &&\\
                    & \displaystyle\sum\limits_{j=1}^{k} C_{j}x_{j} \leq c_{j}
    \end{array}
\end{equation}

\subsection{Market Environment}


\begin{comment}
## PROBLEM OVERVIEW

The central planner’s objective is to minimize the overall system 
operating cost subject to each individual agent’s operational constraints 
(the first set of constraints) and the total shared resources 
capacity constraints (the second set of constraints). 
However, the central planner does not have access to all the relevant information

for decision making so an optimal solution to the central problem 
cannot be directly calculated. The BTM objective is to use a 
market-based resource allocation mechanism to coordinate decentralized 
decision making from agents so that an optimal solution to the central 
problem can be indirectly obtained through an iterative bidding process.

## 

\end{comment}
