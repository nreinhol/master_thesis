\section{The Bundle-Trading-Market Framework}
To start with, we explained in section \ref{sec:lem} that local energy markets enable the trading 
of locally generated energy through a market platform, market mechanism and market access for 
small-scale prosumer and consumer. 
In addition, we presented in section \ref{sec:components_of_local_energy_markets} 
a market mechanism as one of seven core components for a efficient operation of blockchain-based
local energy markets.

On account of that, this chapter covers and explain in detail the applied market mechanism in the 
developed open blockchain-based LEM platform. 

As already briefly introduced in the sections \ref{sec:research_motivation} 
and \ref{sec:Distributed Resource Optimization}, the applied market mechanism is implemented 
through the developed market-based optimization algorithm, which is also called the 
\textit{bundle trading market framework} or short \textit{BTM}. 
In the proposed framework by \shortciteA{guo2012computational}, independent, self-interested
agents trading bundled resources in a double-auction market, which run by a dealer. 
In this case, the dealer replaces the central authority and agents represent distributed
entities. Generally, the \textit{BTM} consists of a master problem that solves 
the market-matching problem, which is managed by the dealer, 
and  subproblems that solve the bundle-determination problem of the dealer.
Further, the solution process is an iteration between the market-matching problem and the 
the bundle-determination problems \shortcite{guo2007market}.
Considering, the \textit{BTM} constitutes a market-based decomposition method 
for decomposable linear systems,
which can be easily implemented to support real-time optimization 
of distributed systems \shortcite{guo2007market}.
Likewise, the central problem of the stated market-based optimization algorithm can 
interpreted as the welfare optimization of all participants in a LEM. 
The dealer, which runs the double auction market, maximizes the welfare through allowing 
agents to trade their preferable bundles of energy.
Therefore, the \textit{BTM} is a suitable market mechanism 
for the concept of a LEM and will be used as such in the developed blockchain-based LEM 
simulation platform. 

\subsection{Problem Overview}
This subsection will give an introductory description to the overall problem. 
With reference to \shortciteA{guo2012computational}, we consider a distributed 
system with $k$ independent agents.
In addition, we have a central problem and individual 
problems of the $k$ individual agents, which both can be expressed as the following 
linear programs. Besides, the \textit{BTM} presented by \shortciteA{guo2007market} requires 
a nondegenerate central problem with a bounded solution. In contrast to discrete markets where 
only integer number of units can be traded, the proposed framework facilitate continuous 
trade amounts. 

\paragraph*{Central Problem}
\begin{equation}
    \begin{array}{ll@{}ll}
        Z(c) = \underset{x_{j} \geq 0}{\text{min}}  & \displaystyle\sum\limits_{j=1}^{k} d^{T}_{j}x_{j} &\\
        \text{s.t.}& \displaystyle N_{j}x_{j} \leq n_{j}, &&i=1 ,..., k  \\
                    & \displaystyle\sum\limits_{j=1}^{k} C_{j}x_{j} \leq c
    \end{array}
\end{equation}

\paragraph*{Agent Problem ($j=1, ..., k$)}
\begin{equation}
    \begin{array}{ll@{}ll}
        z_{j}(c_{j}) =  \underset{x_{j} \geq 0}{\text{min}}  & \displaystyle d^{T}_{j}x_{j} &\\
        \text{s.t.}& \displaystyle N_{j}x_{j} \leq n_{j}, &&\\
                    & \displaystyle\sum\limits_{j=1}^{k} C_{j}x_{j} \leq c_{j}
    \end{array}
\end{equation}

In the table below, the applied variables in the stated equations are described:


\begin{longtable}{c|l}
	\hline
    $d_{j} \in R^{b_{j}}$ & vector of agents $j's$ costs \\
    $x_{j} \in R^{b_{j}}$ & decision variables controlled by agent j \\
    $N_{j} \in R^{a_{j} \times b_{j}}$ & activity matrix \\
    $C_{j} \in R^{m \times b_{j}}$ & activity matrix \\
    $n_{j} \in R^{a_{j}}$ & capacity vector of agent $j's$ independent resources that are managed locally \\
    $c_{j} \in R^{m}$ & agent $j's$ vector of shared resources that can be exchanged with other agents \\
	\hline
	\caption{Applied variables in BTM Framework}
	\label{table:sorted_gas_prices}
\end{longtable} 

Further, the overall objective of the central problem is to minimize the 
operating costs of the overall system. The minimization of the central problem
have to be implemented in consideration of the operational constraints 
(first set of constraints) and the total shared 
resources capacity constraints (second set of constraints) of each individual agent. 
Anyhow, the optimal solution to the central problem cannot be directly calculated, 
due to the lack of access to all relevant information for decision making
($d_{j}, n_{j}, c_{j}, N_{j}, C_{j}, for j=1, ..., k$). 
Additionally, \shortciteA{guo2007market} stated that, for an allocation of 
$c_{j}, j=1, ..., k$ such that $C_{j}x_{j}^{*} \leq c_{j}, \sum\limits_{j=1}^{k} C_{j}$, 
solving the agents’ problems is equivalent to solving the central problem.

Hence, a market-based resource allocation mechanism to coordinate decentralized
decision making from agents is used by the \textit{BTM}. 
For any intially given $c_{j}, j=1, ..., k$, these mechanism enables an indirectly 
acquisition of an optimal solution to the central problem through an direct, iterative and
wealth-improving bidding process among the self-interested agents. 


\subsection{Market Environment}
Generally, a dealer and $k$ independent agents constitute the whole market economy. 
Each of the agent have the opportunity to trade the shared resources $c_{j}$ 
in a double auction market, which is operated by the dealer. Moreover, each agent 
only has local perspective and knowledge, that is to say, they don't know anything about 
the prodcution decisions of the others. 
Further, the market prices to match the trades, will be determined by the dealer.
Subsequently, the roles of the market participants and what they know is described. 
Equally, the market operations like the agent bidding, market matching and settlement 
is outlined and a summary of the market-based algorithm is presented. 

\subsubsection{Agent bidding}
First of all, the initialization takes place. That means, each of the $j=1, ..., k$ agent receives an
endowment of the $m$ shared resources $c_{j}$ , and a cash endowment $e_{j}$.
The wealth of an agent at any point is defined as $e_{j} - z_{j}(c_{j})$ \shortcite{guo2007market}.
Next, agents have the opportunity to buy additional resources or to sell some of their own resources 
to lower their operating costs. 
Therefore, each agent knows the current market prices $p \in R^{m}$ for the shared resources, 
which are published by the dealer. Further, for trading the shared resources in the market, 
a bundle referring to \shortciteA{guo2012computational}, is defined:

\paragraph*{Bundle} A bundle $w \in R^{m}$ is an $m$-dimensional vector of shared resources. 
Each of the $m$ elements corresponds to an amount of one specific shared resource. 
A negative sign of an element signify a sell amount, 
and contrary, a positive sign signify a buy amount. \newline


An \textit{improving bundle set} concerning to lower operating costs is defined as the following:

\paragraph*{Improving Bundle Set}
\begin{equation}
    \begin{array}{ll@{}ll}
        W_{j}(c_{j}|p) = \{w: \exists x_{j} \geq 0 \ni d_{j}^{T}x_{j} + p^{T}w \leq z_{j}(c_{j}), \\
        N_{j}x_{j} \leq n_{j}, C_{j}x_{j} \leq c_{j}+w \}
    \end{array}
\end{equation}

To remember that $z_{j}(c_{j})$ is the optimal value of the agent $j's$ problem 
depending on the amount of the shared resources $c_{j}$. 
That means, an agent looking for bundles that satisfy 
$d_{j}^{T}x_{j} + p^{T}w \leq z_{j}(c_{j})$. If 
$p^{T}w \geq 0$, the term $p^{T}w$ is interpreted as the payment to receive the bundle $w$.
On the opposite, if $p^{T}w \leq 0$, the term $p^{T}w$ constitutes the revenue for selling the bundle $w$.


Accordingly, a rational agent will choose an improving bundle for trading, 
which leaves his wealth level on the same level or better off. 
In addition, the basic market mechanism of \shortciteA{guo2007market} assume no 
strategic actions in the bundle selection and pricing. However, strategic actions 
can be relaxed and \shortciteA{guo2012computational} also incorporate strategic factors
in their extended \textit{BTM}. 
Nevertheless, the applied market mechanism in the developed open blockchain-based
LEM simulation implemented nonstrategic bidding, wherefore an agents bundle selection 
can be defined as the following \textit{bundle determination problem:}

\paragraph*{Bundle Determination Problem}
\begin{equation}
    \begin{array}{ll@{}ll}
        \underset{x_{j} \geq 0, w}{\text{min}}  & \displaystyle d^{T}_{j}x_{j} + p^{T}w &\\
        \text{s.t.}& \displaystyle N_{j}x_{j} \leq n_{j}, &&\\
                    & \displaystyle C_{j}x_{j} \leq c_{j}+w
    \end{array}
\end{equation}

According to \shortciteA{guo2007market}, the \textit{bundle determination problem} have either 
a bounded, or an unbounded solution. A bounded solution is called \textit{limited bundle} 
$w \in R^{m}$, whereby an unbounded solution is called \textit{unlimited bundle}
$u \in R^{m}$.

Furthermore, each agent needs to determine a limit price, that indicates
the maximum an agent is willing to pay for the bundle. 
As already explained above, nonstrategic pricing is implemented. 
Therefore, an agent will always submit a limit price equal to the valuation of the 
bundle. That is to say, $l(w) = v(w)$.

For a \textit{limited bundle} $w$, the value $v(w)$ is defined as:

\begin{equation*}
    \begin{array}{ll@{}ll}
        z_{j}(c_{j}) - z_{j}(c_{j} + w) = z_{j}(c_{j}) - d_{j}^{T}x_{j}.
    \end{array}
\end{equation*}

Whereas for a \textit{unlimited bundle} $u$, the value $v(u)$ describes the 
unit-incremental value that $u$ contributes to the objective change. It is defined
as:

\begin{equation*}
    \begin{array}{ll@{}ll}
        -d_{j}^{T}\hat{x_{j}}.
    \end{array}
\end{equation*}

Potentially, an agent has multiple optimal solutions in 
his bundle-determination problem. Hence, it is allowed to submit more than 
one bundle at a time. 

Finally, old orders and bids of prior rounds without any trades will be treated 
as open orders and bids, because the valuation of an agent for a bundle is
independent of the market price $p$.
It only depends on the respective resource level $c_{j}$. 

\subsubsection{The Dealer’s Market Clearing Mechanism}
First of all, the developed open blockchain-based LEM platform applied a synchronous 
call market where the market prices $p$ are announced periodically by the market dealer. 
However, the synchronous requirement can be relaxed. 
Therefore, \shortciteA{guo2012computational} introduced a ansynchronous market trading 
environment in their e-companion. 

Furthermore, the market has a dealer who trades on her own account. 
The dealer has some initial cash endowment $e_{0}$ and resource endowment $c_{0}$.
Anyway, the sum of all resource endowments needs to fulfil the following equation:

\begin{equation*}
    \begin{array}{ll@{}ll}
        \sum\limits_{j=1}^{k} c_{j} + c_{0} \leq c
    \end{array}
\end{equation*}

Next, agents submit sealed bids to the dealer. The dealer maintains an 
individual order book for each agent. Each individual order book contains 
two different order sets. On the one hand, the set for limited orders $I_{j}$,
on the other hand, the set for unlimited orders $H_{j}$.
If an agent submits orders and no trade is executed, the orders are collected and 
accumulated for the respective agent. Otherwise, any trade from an agent will clear 
the respective order book. 
Lastly, to determine the maximal trade surplus the dealer solves 
the following \textit{market matching problem}:

\paragraph*{Market Matching Problem}

\begin{equation}
    \begin{array}{ll@{}ll}
        \underset{y_{j}^{i} \geq 0, t_{j}^{h} \geq 0 }{\text{max}}  & 
            \sum\limits_{j=1}^{k}
            \left(\sum\limits_{i \in I_{j}}^{} l_{j}(w_{j}^{i})y_{j}^{i} + 
            \sum\limits_{h \in H_{j}}^{} l_{j}(u_{j}^{h})t_{j}^{h} \right) &\\
        \text{s.t.}
            & \sum\limits_{j=1}^{k}
            \left(\sum\limits_{i \in I_{j}}^{} w_{j}^{i}y_{j}^{i} + 
            \sum\limits_{h \in H_{j}}^{} u_{j}^{h}t_{j}^{h} \right) \leq c_{0} \\
            & \sum\limits_{j=1}^{k} y_{j}^{i} \leq 1, \quad j=1 ,..., k
    \end{array}
\end{equation}

The objective of the \textit{market matching problem} maximizes the trade surplus. 

Considering the first set of constraints, if $c_{0} = 0$, it is required that any buy amounts be met by sell amounts. 
On the opposite, if $c_{0} \neq 0$, the dealer supplies additional resources from the inventory
to meet buys. 

With respect to the second set of constraints, it is indicated that the market handles limited and unlimited bundles 
differently. The trades of unlimited bundles in $H_{j}$ are unrestricted, whereas limited bundles in $I_{j}$
are restricted. In that case, limited trades are restricted to be convex combinations of bundles in $I_{j}$.

Besides, the \textit{market matching problem} always has a solution, because setting all variables to zero 
is always a feasible solution. As described by \shortciteA{guo2007market}, the market closes when no trade takes place 
and the market prices remain unchanged. Finally, if the \textit{market matching problem} has a nonzero solution for
 $y_{j}^{i^{*}}$, for $i \in I_{j}$ and $t_{j}^{h^{*}}$, for $h \in H_{j}$, then agent $j$ will have traded the following:

 \begin{equation*}
    \begin{array}{ll@{}ll}
        w_{j}^{*} = 
        \sum\limits_{i \in I_{j}}^{} w_{j}^{i}y_{j}^{i^{*}} + 
        \sum\limits_{h \in H_{j}}^{} u_{j}^{h}t_{j}^{h^{*}}.
    \end{array}
\end{equation*}

Thus, the market clearing prices $p \in R^{m}$ are derived from the dual values of the first set of constraints 
in the \textit{market matching problem}. To remember, the concept of duality is explained in section \ref{sec:duality_theory}.
Consequently, the settlement price for an agent who have traded resources are the following:

\begin{equation*}
    \begin{array}{ll@{}ll}
        p^{T} w_{j}^{*}.
    \end{array}
\end{equation*}

In the following, the whole settlement process is summarized and the calculation of the new values of agent $j$ and the dealer 
is outlined:

\paragraph*{Agent $j$:}
\begin{equation*}
    \begin{array}{ll@{}ll}
        c_{j} \leftarrow c_{j} + w_{j}^{*} \\
        e_{j} \leftarrow e_{j} + p^{T} w_{j}^{*} \\
    \end{array}
\end{equation*}

\paragraph*{Dealer:}
\begin{equation*}
    \begin{array}{ll@{}ll}
        c_{0} \leftarrow c_{0} + \sum\limits_{j=1}^{k} w_{j}^{*} \\
        e_{0} \leftarrow e_{0} + \sum\limits_{j=1}^{k} p^{T} w_{j}^{*} \\
    \end{array}
\end{equation*}


\subsubsection{Summary of Market-Based Algorithm}

Concluding, this subsection will give an overview of the overall, decentralized and synchronous
call market-based algorithm presented by \shortciteA{guo2007market}.

\begin{description}
    \item[Input] \hfill \\
        Initial allocations: $e_{j}, c_{j}$, for $j=1,...,k$, and \\
        initial market prices: $p \geq 0$.
    \item[Output] \hfill \\
        Optimal solutions $x_{j^{*}}$ to the central problem, \\
        optimal allocations $c_{j^{*}}$, for $j=1,...,k$, and \\
        optimal market prices $p^{*}$.
    \item[Step 0] \hfill \\
        Set $I_{j} = \emptyset, H_{j} = \emptyset$, for $j=1,...,k$, and \\
        the last market-price vector $\pi = 0$.
    \item[Step 1] \hfill \\
        The dealer saves the current market prices $\pi \leftarrow p$. Each agent solves his 
        \textit{bundle determination problem} and adds new limited bundle orders to $I_{j}$ and 
        unlimited bundle orders to $H_{j}$.
    \item[Step 2] \hfill \\
        The dealer solves the \textit{market matching problem} ($w_{j}^{*}$ represents a matched bundle for agent $j$). 
    \item[Step 3] \hfill \\
        If no nonzero solution exists, this step will be skipped and continued with \textit{Step 4}.
        If a nonzero solution to the \textit{market matching problem} exists and the trades includes items from at least
        one agent and the dealer, the settlement takes place. 
        The dealer publishes the shadow prices $p$ as the new market prices.
        The new allocations are calculated as follows:
        $c_{j} \leftarrow c_{j} + w_{j}^{*}$, for $j=1,...,k$ and $c_{0} \leftarrow c_{0} + \sum\limits_{j=1}^{k} w_{j}^{*}$.
        Further, the new endowments are calculated as follows:
        $e_{j} \leftarrow e_{j} + p^{T} w_{j}^{*}$ for $j=1,...,k$ and $e_{0} \leftarrow e_{0} + \sum\limits_{j=1}^{k} p^{T} w_{j}^{*}$.
        Reset $I_{j} =  \emptyset$ and $H_{j} = \emptyset$ for each agent $j$ with $w_{j}^{*} \neq 0$.
        Now, go to \textit{Step 1}. 
    \item[Step 4] \hfill \\
        The dual values of the clearing constraints from the \textit{market matching problem} form the new market prices $p$. 
        If the new market prices $p \neq \pi$ , go to \textit{Step 1}, otherweise, stop.
\end{description}

\begin{figure}[htbp]
	\centering
	\includegraphics[width=.55\linewidth]{./figures/btm_flowchart.png}
	\caption{Flowchart of the BTM Framework}
	\label{figure:node_network}
\end{figure}




\begin{comment}

\end{comment}
