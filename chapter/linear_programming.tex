% everything of the following originate from the book 'Market Design' from Martin Bichler

\section{Linear Programming}
Linear optimization refers to the problem of optimizing a linear objective function with the
decision variables , $x_{1}, ..., x_{n}$ subject to linear equality or inequality constraints. 
In canonical form a linear programm (LP) is given as:

\begin{equation}
    \begin{array}{ll@{}ll}
        \text{max}  & \displaystyle\sum\limits_{j=1}^{n} c_{j}x_{j} &\\
        \text{s.t.}& \displaystyle\sum\limits_{j=1}^{n} a_{ij}x_{j} \leq b_{i},  &&i=1 ,..., m\\
                    &                        x_{j} \geq 0, &&j=1 ,..., n
    \end{array}
\end{equation}

\subsection{Duality Theory}

\section{Market Design}


\textit{Mathematical Optimization} plays a central role in the design and analysis of multi-object 
markets and also provides a central foundation for theoretical models of mulit-object markets. 

However, market design is more than mathematical programming, as it takes into consideration bidding
strategies and human behavior. Ignoring bidding strategies and different types of manipulation 
in market-based resource allocation is like optimizing a problem with the wrong parameters. 

Overall, makret design aims at design principles and rules for market institutions which are robust
against manupulation and allow bidders to express their preferences, so that the designer can aim 
for good or even optimal allocations. This is similar to network security, where designers aim 
for secure network protocols that are hard to tamper with, 
knowing that there is no absolute security

\subsubsection{Closed Order Book}