% everything of the following originate from the book 'Market Design' from Martin Bichler

\section{Linear Programming}
\label{sec:linear_progamming}
This section will give an introduction into the basics of mathematical linear programming
and linear programming duality.
It will present the purpose and necessity of optimization models and will give
an example how an optimization model is constructed. 

To begin with, optimization models try to define in mathematical terms, the goal
of solving a problem in the best or optimal way. 
This can be applied in many different areas, for example, it might mean running a business to 
maximize profit, designing a bridge to minimize weight or selecting a flight plan for an aircraft
to minimize time or fuel use \shortcite{griva2009linear}. 
The use case of solving an optimization problem optimally is so ubiquitous, 
that optimization models are used in almost every area of application \shortcite{griva2009linear}.
Often it is not possible or economically feasible to make decisions without the help
of such a model. Due to the excellent improvements in computer hardware and software
in the last decades, optimization models became a practical tool 
in business, science and engineering \shortcite{griva2009linear}. 
Finally, it is possible to solve optimization problems with a huge set of variables. 

Further, linear optimization, also known as a \textit{linear program}, refers to the optimization
of a linear function with the decision variables, $x_{1}, ..., x_{n}$ subject to 
linear equality or inequality constraints. 
As presented by \shortciteA{bichler2017market}, in canonical form 
a linear optimization model is given as:

\begin{equation*}
    \begin{array}{ll@{}ll}
        \text{max}  & \displaystyle\sum\limits_{j=1}^{n} c_{j}x_{j} &\\
        \text{s.t.}& \displaystyle\sum\limits_{j=1}^{n} a_{ij}x_{j} \leq b_{i},  &&i=1 ,..., m\\
                    &                        x_{j} \geq 0, &&j=1 ,..., n
    \end{array}
\end{equation*}

However, a linear program can be modeled in different forms. It is also possible 
to write a model in the matrix-vector notation. To represent the above model in this form, letting
$x=(x_{1}, ..., x_{n})^{T}$, $c=(c_{1}, ..., c_{n})^{T}$, $b=(b_{1}, ..., b_{m})^{T}$
and name the matrix of the coefficients $a_{ij}$ by $A$. As introduced by \shortciteA{griva2009linear},
the model becomes:

\begin{equation*}
    \begin{array}{ll@{}ll}
        \text{max}  & \displaystyle c^{T}x &\\
        \text{s.t.}& \displaystyle Ax \leq b&\\
                    &                        x \geq 0
    \end{array}
\end{equation*}

\clearpage
In general, the objective function of a linear model is either minimized or maximized.
Additionally, the constraints may include a combination of inequalities and equalities
and the variables are unrestricted or restricted in sign \shortcite{bichler2017market}.

\subsection{Duality Theory}
\label{sec:duality_theory}
In addition, for every linear program exists another problem, which
is called the \textit{dual} of the original linear problem, also known as the 
\textit{primal} \shortcite{bichler2017market}.
In the \textit{dual}, the roles of variables and constraints are reversed. 
That means, every variable of the \textit{primal} becomes a constraint in the \textit{dual},
and every constraint in the \textit{primal} becomes a variable in 
the \textit{dual} \shortcite{griva2009linear}.
For instance, if the \textit{primal} has $n$ variables and $m$ constraints, the \textit{dual}
will have $m$ variables and $n$ constraints. Further, the \textit{primal} objective coefficients 
are the coefficients on the right-hand side of 
the \textit{dual}, and vice versa.
Finally, the transposed coefficient matrix of the \textit{primal} constitutes the matrix in the 
\textit{dual} \shortcite{griva2009linear}.
To give an example of a \textit{primal} and the corresponding \textit{dual} linear program,
the representation by \shortciteA{bichler2017market} is considered:

\begin{equation}
    \tag{primal}
    \begin{array}{ll@{}ll}
        \text{min}  & \displaystyle c^{T}x &\\
        \text{s.t.}& \displaystyle Ax \geq b  &\\
                    &                        x \geq 0
    \end{array}
\end{equation}

\begin{equation}
    \tag{dual}
    \begin{array}{ll@{}ll}
        \text{max}  & \displaystyle b^{T}y &\\
        \text{s.t.}& \displaystyle A^{T}y \leq c &\\
                    &                        y \geq 0
    \end{array}
\end{equation}


Next, the motivation of the duality theory can explained by the issue of trying to find 
bounds on the value of the optimal solution to a linear programming problem. 
If the \textit{primal} is a minimization problem, the \textit{dual} represents a lower bound
on the value of the optimal solution, otherwise, if the \textit{primal} is a maximization problem,
the \textit{dual} represents an upper bound \shortcite{bichler2017market}.
To illustrate the concept and motivation of duality theory more tangible, an example stated by 
\shortciteA{bichler2017market} is given. 
In an application, the variables in the \textit{primal} linear problem stand for consumer products and
the objective coefficients represent the profits linked to the production of those consumer products. 
In this case, the objective in the \textit{primal} describes directly the relation between a change
in the production of the products and profit. Whereas, the constraints in the \textit{primal} 
describe the availability of raw materials. Consequently, an increase in the availability of 
raw materials needed for the production of the consumer products, enables an increase in production,
and finally an increase in the overall profit. 
However, the \textit{primal} problem does not outline this relationship reasonable. Hence, one of 
the benefit of the \textit{dual} is to properly show the effect of changes in the constraints on the
value of the objective. Due to this, the variables in a \textit{dual} linear program often called 
\textit{shadow prices}, since they describe the hidden costs associated with the constraints. 

\subsubsection{Weak Duality Theorem}
As stated in the section before (\ref{sec:duality_theory}), the \textit{dual} problem 
provides, depending on maximization or minimization problem, lower and upper bounds for 
the \textit{primal} objective function. Referring to \shortciteA{vanderbei2015linear},
this result is true in general and is described as the \textit{Weak Duality Theorem}:

\begin{theorem}
    If $x$ is feasible for the primal and 
    $y$ is feasible for the dual, then:

    \begin{equation*}
        \begin{array}{ll@{}ll}
            \displaystyle c^{T}x \leq b^{T}y&\\
        \end{array}
    \end{equation*}    

\end{theorem}

Regarding to \shortciteA{griva2009linear}, there are some logical corollaries 
of the \textit{Weak Duality Theorem}:

\begin{corollary}
    If the primal is unbounded, then the dual is infeasible. 
    If the dual is unbounded, then the primal is infeasible.
\end{corollary}

\begin{corollary}
\label{strong_duality_corollary}
    If $x$ is a feasible solution to the primal, 
    $y$ is a feasible solution to the dual,
    and $c^{T}x = b^{T}y$, then x and y are optimal for their respective problems.
\end{corollary}

\subsubsection{Strong Duality Theorem}
Concerning to corollary \ref{strong_duality_corollary}, it is possible to verify if 
the points x and y are optimal solutions, without solving the corresponding linear programs. 
Furthermore, this corolary is also used in the proof of strong duality \shortcite{griva2009linear}. 
The \textit{Strong Duality Theorem} describes, that for linear programming there 
is never a gap between the \textit{primal} and the \textit{dual} 
optimal objective values \shortcite{vanderbei2015linear}.

\begin{theorem}
    If  the primal problem has an optimal solution $x^{*}$,
    then the dual also has an optimal solution $y^{*}$,
    such that:

    \begin{equation*}
        \begin{array}{ll@{}ll}
            \displaystyle c^{T}x^{*} = b^{T}y^{*}&\\
        \end{array}
    \end{equation*}    

\end{theorem}

\subsubsection{Simplex-Method}
This subsection will give a brief introduction to the \textit{simplex-method}. 
To start with, in the field of linear programming, the \textit{simplex-method} 
is the most widely used algorithm. 
This method was developed around 1940 and had several competitors at that time.
However, the algorithm was able to assert itself due to its efficiency. \shortcite{griva2009linear}
In general, the \textit{simplex-method} is an iterative algorithm, used to solve 
linear programming models \shortcite{griva2009linear}. To apply the algorithm to a model, 
it must be in standard form \shortcite{Luenberger2008}. According to \shortciteA{griva2009linear}, 
we consider the following linear progamming model:

\begin{equation*}
    \begin{array}{ll@{}ll}
        \text{min}  & \displaystyle z = -x_{1} - 2x_{2} &\\
        \text{s.t.}& \displaystyle - 2x_{1} + x_{2} \leq 2 &\\
                    & \displaystyle - x_{1} + 2x_{2} \leq 7 &\\
                    & \displaystyle  x_{1} \leq 3 &\\
                    &                        x_{1}, x_{2} \geq 0
    \end{array}
\end{equation*}

Next, we give an example to illustrate how to convert a linear program into standard form.
To achieve that, the so called \textit{slack variables} will be added, which represents the 
difference between the right-hand side and the left-hand side. 

\begin{equation*}
    \begin{array}{ll@{}ll}
        \text{min}  & \displaystyle z = -x_{1} - 2x_{2} &\\
        \text{s.t.}& \displaystyle - 2x_{1} + x_{2} + x_{3} = 2 &\\
                    & \displaystyle - x_{1} + 2x_{2} + x_{4} = 7 &\\
                    & \displaystyle  x_{1} + x_{5} = 3 &\\
                    &                        x_{1}, x_{2}, + x_{3}, + x_{4}, + x_{5} \geq 0
    \end{array}
\end{equation*}


Finally, with the linear progamming model in standard form, the idea of the \textit{simplex-method}
is to iteratively proceed from one basic feasible solution to another \shortcite{Luenberger2008}.
Thereby, it always looks for a feasible solution which is better then the one before. 
The definition of better in this case depends on the nature of the linear problem. 
The process is performed until a solution is reached which cannot 
be improved anymore. In the end, this final solution is then an optimal solution \shortcite{vanderbei2015linear}. 

\begin{comment}



# market design
Mathematical Optimization plays a central role in the design and analysis of multi-object 
markets and also provides a central foundation for theoretical models of mulit-object markets. 

However, market design is more than mathematical programming, as it takes into consideration bidding
strategies and human behavior. Ignoring bidding strategies and different types of manipulation 
in market-based resource allocation is like optimizing a problem with the wrong parameters. 

Overall, makret design aims at design principles and rules for market institutions which are robust
against manupulation and allow bidders to express their preferences, so that the designer can aim 
for good or even optimal allocations. This is similar to network security, where designers aim 
for secure network protocols that are hard to tamper with, 
knowing that there is no absolute security

# closed order book

\end{comment}



