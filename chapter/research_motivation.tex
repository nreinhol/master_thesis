\section{Research Motivation}
\label{sec:research_motivation}

% RES, explanation and so
The generation from distributed renewable energy sources (RES) is constantly increasing \shortcite{mengelkamp2018designing}. In contrast to power plants which run by non-renewable fossil fuels, distributed RES produce energy in a decentralized and volatile way, which is hard to predict. These characteristics of the distributed RES challenge the current energy system \shortcite{ampatzis2014local}.

% the current electric grid and the problems
The existing electric grid is build for centralized generation by large power plants and the design of the current wholesale markets is not able to react in real-time to a significant amount of distributed RES \shortcite{mengelkamp2018designing} \shortcite{ampatzis2014local}. Moreover, this way of energy generation is economically not ideal because of energy losses due to long physical distances between generation and consumption parties. 

% introduction of p2p energy trading
Therefore, new market approaches are needed, to successfully integrate the increasing amount of distributed RES \shortcite{mengelkamp2018blockchain}. A possible solution to the technical and market problems is Peer-to-Peer (P2P) energy trading in local energy markets (LEM) \shortcite{long2017feasibility}. 

% explanation of local energy markets
LEM, also called microgrid energy markets, consist of small scale prosumers, consumers and a market platform which enables the trading of locally generated energy between the parties of a community. Due to the trading of locally generated energy within the related communities, LEM support sustainability and an efficient use of distributed renewable energy sources. Likewise, the need of expensive and inefficient transportation of energy through long physical distances can be reduced. The concept of LEM strengthens the self-sufficiency of communities and enables possible energy cost reductions. Moreover, profits remain within the communities whereby reinvestments in additional RES are promoted \shortcite{mengelkamp2018designing}. 

% introduction of blockchain as underlying technology 
However, P2P energy trading in LEM requires advanced communication and data exchanges between the different parties, which makes central management and operation more and more challenging. The implementation of LEM needs local distributed control and management techniques. \shortcite{andoni2019blockchain}. Therefore, a new and innovative information communication technology (ICT) is required.  A possible solution provides the emerging distributed ledger technology (DLT). It is designed to enable distributed transactions without a central trusted entity. Furthermore, an Ethereum-based blockchain allows the automated execution of smart contracts depending on vesting conditions, which suits the need of LEM for decentralized and autonomous market mechanisms. This offers new approaches and market designs. Accordingly, DLT can help addressing the challenges faced by decentralized energy systems. However, DLTs are not a matured technology yet and there are several barriers in using them, especially for researcher who do not have a technically background. 

% introduce the topic of competitive benchmarking
In addition, Ketter, Peters, Collins and Gupta introduce the approach of Competitive Benchmarking (CB) \shortcite{ketter2015competitive}. This approach describes a research method which faces a real world wicked problem that is beyond the capacity of a single discipline. This is realized by developing a shared paradigm which is represented in a concrete open simulation platform. In detail, it consists of the three principal elements \textit{CB Alignment}, \textit{CB Platform} and \textit{CB Process}. The \textit{CB Alignment} refers to the constant synchronization process between the shared paradigm and the wicked problem. Further, the \textit{CB Platform} represents the medium, in which the shared paradigm is technically illustrated. In addition, the \textit{CB Platform} provides the infrastructure for the third element \textit{CB Process}. It describes the iterative development of new theories and design artifacts through independent researchers, which influence each other and improving their work in direct sight of each other. 

% explain why competitive benchmarking is useful
Due to the plurality of involved parties and the interdisciplinary requirements for the implementation of new energy market approaches, the accessibility is of major importance.
Therefore, the presence of such a open simulation platform depicting the shared paradigm of a LEM, would ensure the accessibility and could help to gain new valuable outcomes in the research field of LEM or new energy market designs in general. 

% introduction of optimization decomposition algorithm
Further, Guo, Koehler and Whinston developed a market-based optimization algorithm, which solves a distributed system optimization problem by self-interested agents iteratively trading bundled resources in a double auction market run by a dealer. The authors called it bundle trading market framework or short BTM \shortcite{guo2007market}. The central problem of the stated market-based optimization algorithm can interpreted as the welfare optimization of all participants in a LEM. The dealer, which runs the double auction market, maximizes the welfare through allowing agents to trade their preferable bundles of energy. Hence, the stated BTM implemented on basis of a blockchain as underlying ICT can depict the concept of a LEM.

% bringing all together
Consequently, this research will bring all the introduced approaches together and develop an open blockchain-based LEM simulation, which enables the research approach based on the three stated elements of CB. The platform will be realized through the introduced optimization algorithm with a blockchain as the underlying ICT. That means, the smart contract takes the role of the market dealer and the self-interested agents represent the individual participants in a LEM. The focus of this paper will be on the implementation and software design of the open blockchain-based LEM simulation platform.

\clearpage