\section{Research Motivation}
\label{sec:research_motivation}

% RES, explanation and so
The generation from distributed \textit{renewable energy sources (RES)} is constantly increasing \shortcite{mengelkamp2018designing}. 
In contrast to power plants which run by non-renewable fossil fuels, distributed RES produce energy in a decentralized and volatile way, which is hard to predict. 
These characteristics of the distributed RES challenge the current energy system \shortcite{ampatzis2014local}.

% the current electric grid and the problems
The existing electric grid is built for centralized generation by large power plants 
and the design of the current wholesale markets
is not able to react in real-time to a significant amount of distributed RES \shortcite{mengelkamp2018designing} \shortcite{ampatzis2014local}. 
Moreover, this way of energy generation is economically not ideal because of energy losses due to long physical
distances between generation and consumption parties. 

% introduction of p2p energy trading
Therefore, new market approaches are needed, to successfully integrate the increasing amount of distributed RES \shortcite{mengelkamp2018blockchain}. 
A possible solution to the technical and market problems is \textit{Peer-to-Peer (P2P)} energy trading in \textit{local energy markets (LEM)} \shortcite{long2017feasibility}. 

% explanation of local energy markets
LEM, also called microgrid energy markets, consist of small scale prosumers, consumers and a market platform that enables the trading 
of locally generated energy between the parties of a community.
Due to the trading of locally generated energy within the related communities,
LEM support sustainability and the efficient use of distributed RES.
Likewise, the need for expensive and inefficient transportation of energy through long physical 
distances can be reduced. The concept of LEM strengthens the self-sufficiency of communities and 
enables possible energy cost reductions. Moreover, profits remain within the communities 
whereby reinvestments in additional RES are promoted \shortcite{mengelkamp2018designing}. 

% introduction of blockchain as underlying technology 
However, P2P energy trading in LEM requires advanced communication and data exchanges between the different parties, 
which makes central management and operation more and more challenging. The implementation of LEM needs local 
distributed control and management techniques \shortcite{andoni2019blockchain}. 
Therefore, a new and innovative \textit{information communication technology (ICT)} is required.  
The emerging \textit{distributed ledger technology (DLT)} provides a possible solution. 
It is designed to enable distributed transactions without a central trusted entity. 
Furthermore, an Ethereum-based blockchain allows the automated execution of smart contracts depending 
on vesting conditions, which suits the need of LEM for decentralized and autonomous market mechanisms. 
This offers new approaches and market designs. Accordingly, DLT can help to address the challenges 
faced by decentralized energy systems. However, DLTs are not a matured technology yet and 
there are several barriers in using them, especially for the researcher who do not have a technical background. 

% introduce the topic of competitive benchmarking
In addition, \shortciteA{ketter2015competitive} introduce the approach of \textit{Competitive Benchmarking (CB)}. 
This approach describes a research method that faces a real-world wicked problem that is beyond the capacity of a single discipline. 
It is realized by developing a shared paradigm that is represented in a concrete open simulation platform. 
In detail, it consists of the three principal elements \textit{CB Alignment}, \textit{CB Platform} and \textit{CB Process}. 
The CB Alignment refers to the constant synchronization process between the shared paradigm and the wicked problem. 
Further, the CB Platform represents the medium, in which the shared paradigm is technically illustrated. 
In addition, the CB Platform provides the infrastructure for the third element CB Process. 
It describes the iterative development of new theories and design artifacts through independent researchers, 
who influence each other and improve their work in direct sight of each other. 

% explain why competitive benchmarking is useful
Due to the plurality of involved parties and the interdisciplinary requirements for the 
implementation of new energy market approaches, the accessibility is of major importance.
Therefore, the presence of such an open simulation platform depicting the shared paradigm of LEM, 
would ensure the accessibility and could help to gain new valuable outcomes in the research field 
of LEM or new energy market designs in general. 

% introduction of optimization decomposition algorithm
Further, \shortciteA{guo2007market} developed a market-based optimization algorithm
that solves a distributed system optimization problem by self-interested agents iteratively 
trading bundled resources in a double auction market run by a dealer. The authors called it 
\textit{bundle trading market framework (BTM)}. The central problem of the stated market-based 
optimization algorithm can be interpreted as the welfare optimization of all participants in LEM. 
The dealer, which runs the double auction market, maximizes the welfare through allowing agents 
to trade their preferred bundles of energy. Hence, the stated BTM implemented on the basis of a 
blockchain as underlying ICT can depict the concept of LEM.

% bringing all together
Consequently, this research brings all the introduced approaches together 
and develops an open blockchain-based LEM simulation, which enables the research 
approach based on the three stated elements of CB. The platform is realized through 
the introduced optimization algorithm with a blockchain as the underlying ICT. 
That means, the smart contract takes the role of the market dealer and the self-interested 
agents represent the individual participants. The focus of this paper is on the 
implementation and software design of the open blockchain-based LEM simulation platform.

\clearpage