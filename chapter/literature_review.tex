\subsection{Blockchain-based Energy Markets}
\label{sec:Blockchain-based Energy Markets}
To start off, Mihaylov et al. were the first who adressed blockchain technology in energy markets. They present a new decentralized digital currency with the aid of which prosumers trade locally produced renewable energy \cite{mihaylov2014nrgcoin}. In their introduced concept, the generation and consumption of renewable energy is direct transferable into virtual coins. However, the market value of the virtual currency is determined centrally by the distributed system operator. Further, Al Kawasmi et al. introduce a blockchain-based model for a decentralized carbon emission trading infrastructure \cite{al2015bitcoin}. Their model based on the bitcoin protocol  and focus on privacy and system security goals. Besides, they provide a solution to the problem of anonymous carbon emission trading. Equally, Aitzhan and Svetinovic address the issue of transaction security in decentralized smart grid energy trading and implemented a proof-of-concept for a blockchain-based energy trading system including anonymous encrypted messaging streams \cite{aitzhan2018security}. Concluding, they show that blockchains enable the implementation of decentralized energy trading and that the degree of privacy and security is higher than in traditional centralized trading platforms. Furthermore, Sikorski et al. present an proof-of-concept where a blockchain enables machine-to-machine (M2M) interactions depicting an M2M energy market \cite{sikorski2017blockchain}. They pointed out that the blockchain technology has significant potential to support and enhance the 4th industrial revolution. Moreover, Mengelkamp et al. reveal the concept of a blockchain-based local energy market without the need of a thrusted third entity \cite{mengelkamp2018designing}. In addition, they deduce seven market components as a framework for building efficient microgrid energy markets. Consequently, the Brooklyn Microgrid project is introduced and evaluated according the market components. As a result, the Brooklyn Microgrid also shows that blockchains are a suitable technology to implement decentralized microgrid energy markets, though current regulation does not allow to run such a local energy markets in most of the countries. Later on, Mengelkamp et al. present an initial proof-of-concept of a simple blockchain-based concept, market design and simulation of a local energy market consisting of hundred households \cite{mengelkamp2018blockchain}. Finally, they conclude that the real-life realisation and technological limitations of such blockchain-based market approaches need to be investigated by further research. In addition, it is mentioned that regulatory changes will play an important role in the future of blockchain-based LEM.

\subsection{Distributed Resource Optimization}
\label{sec:Distributed Resource Optimization}
To begin with, Fan et al. outline a new approach for the development of an information system which can be used for the problem of a supply chain. The concept demonstrate a decentralized decision making process that is realized through the design of a market-based coordination system which incite the pariticipants to act in a way that is beneficial to the overall systems \cite{fan2003decentralized}. Further, Guo et al. revive this concept and develop a market-based decomposition method for decomposable linear systems, that can be easily implemented to support real-time optimization of dsitributed systems \cite{guo2007market}. They prove that the system optimality can be achieved under a dynamic market-trading algorithm in a finite number of trades. Moreover, the outlined alogrithm can be operated in synchronous and as well in asynchronous environments. Later on, Guo et al. extend their stated concept to a dynamic, asynchronous internet market environment \cite{guo2012computational}. Additionally, they examine how various market design factors like dealer inventory policies, market communication patterns, and agent learning strategies affect the computational market efficiency and implementation. Finally, also this time, Guo et al. prove finite convergence to an optimal solution under all these different schemes. 

\subsection{Competitive Benchmarking}
Firstly, March and Smith describe a two dimensional framework for research in IS \cite{march1995design}. The two dimensions can be distinguished into behavioral-science and design-science. The behavioral-science paradigm is based on explaining or predicting human or organizational behavior. Whereas, the design-science paradigm is based on broad types of outputs produced by design research. It strives to extend the boundaries of human and organizational capabilities through the creation of new artifacts. 

% connection of these two - design science approach by hevner
Furthermore, the research framework presented by Hevner et al. combines the both stated IT research paradigms and illustrates the interaction between these two \cite{hevner2008design}. They present that the technology and behavior are inseparable in an information system. Therefore, they argue that considering the complementary research cycle between design-science and behavioral-science is crucial to address fundamental problems faced in the productive application of information technology.        

% competitive benchmarking include basic principles of hevner
In addition, Ketter, Peters, Collins and Gupta introduce the IS research approach called Competitive Benchmarking (CB), which includes varios basic principles of the design science approach of Hevner et al. \cite{ketter2015competitive}. This IS research concept is designed for so called wicked problems and address the problems and needs of interdisciplinary research communities. CB focus on the interconnection of problems at the same and different levels to imitate real world. In repeated competitions, in which individual teams compete against each other, the developed environment and models can be tested and evaluated. This results in a diversity of outcoming designs, which are difficult to achieve in traditional design science frameworks.  

