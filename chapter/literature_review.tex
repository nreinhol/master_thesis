\section{Literature Review}
\label{sec:literature_review}

\subsection{Blockchain-based Energy Markets}
\label{sec:Blockchain-based Energy Markets}
To start, \shortciteA{mihaylov2014nrgcoin} were the first who addressed blockchain technology in
energy markets. They presented a new decentralized digital currency with the aid of
which prosumers can trade locally produced renewable energy.
In their introduced concept, the generation and consumption of renewable energy are directly
transferable into virtual coins. However, the market value of the virtual currency is determined
centrally by the distributed system operator. Further, \shortciteA{al2015bitcoin} introduced a blockchain-based
model for a decentralized carbon emission trading infrastructure.
Their model based on the bitcoin protocol and focus on privacy and system security goals.
Besides, they provide a solution to the problem of anonymous carbon emission trading.
Equally, \shortciteA{aitzhan2018security} addressed the issue of transaction security in
decentralized smart grid energy trading and implemented a proof-of-concept for a
blockchain-based energy trading system including anonymous encrypted messaging streams.
Concluding, they have shown that blockchains enable the implementation of decentralized energy trading and that the degree
of privacy and security is higher than in traditional centralized trading platforms. 
Furthermore, \shortciteA{sikorski2017blockchain}
presented a proof-of-concept where a blockchain enables machine-to-machine (M2M) interactions depicting an
M2M energy market. They pointed out that the blockchain technology has
significant potential to support and enhance the 4th industrial revolution. 
Moreover, \shortciteA{mengelkamp2018designing}
revealed the concept of a blockchain-based local energy market without the need of a thrusted
third entity. In addition, they deduced seven market
components as a framework for building efficient microgrid energy markets. Consequently,
the Brooklyn Microgrid project is introduced and evaluated according to the market components.
As a result, the Brooklyn Microgrid has also shown that blockchains are a suitable technology
to implement decentralized microgrid energy markets, though current regulation does not
allow running such LEM in most of the countries. Later on, \shortciteA{mengelkamp2018blockchain}
presented a initial proof-of-concept of a simple blockchain-based concept, market design and
simulation of a local energy market consisting of a hundred households.
Finally, they concluded that the real-life realization and technological limitations of such blockchain-based
market approaches need to be investigated by further research. 
In addition, it is mentioned that regulatory
changes will play an important role in the future of blockchain-based LEM.

\subsection{Distributed Resource Optimization}
\label{sec:Distributed Resource Optimization}
To begin with, \shortciteA{fan2003decentralized} outlined a new approach for the 
development of an information system
that can be used for the problem of a supply chain. The concept demonstrated a decentralized
decision-making process that is realized through the design of a market-based coordination
system which incites the participants to act in a way that is beneficial to the overall
systems. Further, \shortciteA{guo2007market} revived this concept
and developed the BTM, a market-based decomposition method for decomposable linear systems,
which can be easily implemented to support real-time optimization of distributed
systems. They proved that the system optimality can be
achieved under a dynamic market-trading algorithm in a finite number of trades.
Moreover, the outlined algorithm can be operated in synchronous and as well in
asynchronous environments. Later on, \shortciteA{guo2012computational} extended 
their stated concept to a dynamic, asynchronous internet market environment.
Additionally, they examined how various market design factors like dealer inventory
policies, market communication patterns, and agent learning strategies affect
computational market efficiency and implementation. Finally,
they proved finite convergence to an optimal solution under all these different schemes.

\subsection{Competitive Benchmarking}
Firstly, \shortciteA{march1995design} described a two-dimensional framework for research in IS.
The two dimensions can be distinguished into behavioral-science and design-science. The behavioral-science
paradigm is based on explaining and predicting human or organizational behavior. Whereas, the design-science
paradigm is based on broad types of outputs produced by design research. It strives to extend the boundaries
of human and organizational capabilities through the creation of new artifacts.

% connection of these two - design science approach by hevner
Furthermore, the research framework presented by \shortciteA{hevner2008design} combines both stated
IT research paradigms and illustrates the interaction between these two.
They presented that technology and behavior are inseparable in an information system. Therefore,
they argued that considering the complementary research cycle between design-science and behavioral-science
is crucial to address fundamental problems faced in the productive application of information technology.

% competitive benchmarking include basic principles of hevner
In addition, \shortciteA{ketter2015competitive} introduced the IS research approach CB,
which includes various basic principles of the design science approach of \shortciteA{hevner2008design}.
This IS research concept is designed for so-called wicked problems and addresses the problems and needs of interdisciplinary
research communities. CB focuses on the interconnection of problems at the same and different levels to imitate the real world.
In repeated competitions, in which individual teams compete against each other, the developed environment and models
can be tested and evaluated. This results in a diversity of outcoming designs, which are difficult to achieve in
traditional design science frameworks.

\clearpage